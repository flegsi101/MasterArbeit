\section{Implementierung}

\subsection{Allgemeines Vorgehen}
Für die Implementierung soll das existierende System erweitert werden, so dass vorläufig die alte Funktionalität erhalten bleibt.
Da zu wird im ersten Schritt die URL zur REST-API angepasst. 
Die alte API soll danach und dem Pfad \"/api/v1\" und die neue unter \"api/v2\" erreichbar sein.
Der alte Web-Client muss dann dementsprechend angepasst werden und wird auch auf einen anderen Pfad \"/v1\" verschoben.
Der neue Client soll dann sowohl unter keinem Pfad \"/\" als auch unter \"/v2\", um einheitlich zu bleiben, erreichbar sein.

Da bereits ein Server entwickelt wurde, der eine REST-API bietet, soll dieser auch weiter benutzt werden und nur um die neuen Funktionalitäten erweitert werden.
Daher ist dieser auch für das zuordnen der Pfade auf die Endpunkte zuständig.
Da es für den Web-Client, im Gegensatz zur API, keine weiteren Abhängigkeiten geben kann, kann die Ingestion direkt auf die neue API angepasst werden.