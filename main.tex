% ||================================================================================================
% || Präambel
% ||================================================================================================

% |=================================================================================================
% | Layout
% |=================================================================================================
\documentclass[12pt,titlepage]{report}
% digital
% print
%\documentclass[12pt,titlepage]{book}
\usepackage{geometry}
\geometry{
  left=3cm,
  right=3cm,
  % print
  % bindingoffset=5mm
}

% |=================================================================================================
% | Fonts
% |=================================================================================================
\usepackage[onehalfspacing]{setspace}
\usepackage{dejavu} 
\usepackage{sectsty}
\allsectionsfont{\sffamily}
\usepackage[T1]{fontenc}
\usepackage[utf8]{inputenc} % direkte Einbgabe von Umlauten

% language stuff
\usepackage[german]{babel}

% miscellaneous
\usepackage{graphicx, subfigure}         % graphics
\graphicspath{{grafiken/}}
\usepackage[german]{fancyref}
\usepackage{hhline}           % double lines in tables
\usepackage{amsfonts}         % real numbers etc.
\usepackage[rightcaption]{sidecap} % figure captions on the right (optional)
\usepackage{hyperref}         % for URLs
\usepackage{listings}         % for code samples
\usepackage{fancyhdr}         % for header line

\usepackage[backend=biber, style=authoryear-icomp]{biblatex}
\addbibresource{literatur.bib}
% \usepackage{natbib}
% Hier bei Bedarf die Seitenränder einstellen
\usepackage[german]{algorithm2e}
\RestyleAlgo{ruled}

\usepackage[german]{cleveref}

\usepackage[printonlyused]{acronym}
\usepackage{multirow, multicol, tabularx}
\usepackage{csquotes}
\newcommand*{\fancyrefalgolabelprefix}{algo}

\newcommand*{\Frefalgoname}{ALGORITHMUS}
\newcommand*{\frefalgoname}{Algorithmus}

\frefformat{plain}{\fancyrefalgolabelprefix}{\frefalgoname\fancyrefdefaultspacing#1}
\Frefformat{plain}{\fancyrefalgolabelprefix}{\Frefalgoname\fancyrefdefaultspacing#1}

\frefformat{vario}{\fancyrefalgolabelprefix}{\frefalgoname\fancyrefdefaultspacing#1#3}
\Frefformat{vario}{\fancyrefalgolabelprefix}{\Frefalgoname\fancyrefdefaultspacing#1#3}

% Kopf- und Fußzeile
\fancyhead{} % clear all header fields
%\fancyhead[RO,LE]{\leftmark}
\fancyhead[RO]{\leftmark}
\setlength{\headheight}{15pt}

\title{Arbeit}
\author{Alexander Martin}
\date{August 2021}

\begin{document}

\begin{titlepage}
  \begin{center}
    {\Large\bf Entwurf und Implementierung einer generischen Ingestion-Schnittstelle mit Versionierung für Data-Lake-Systeme}\\[2cm]

    {\bf Masterarbeit}\\
    zur Erlangung des Grades {\em Master of Science}\\[1cm]

    an der\\
    Hochschule Niederrhein\\
    Fachbereich Elektrotechnik und Informatik\\
    Studiengang {\em Informatik}\\[2cm]

    vorgelegt von\\
    Alexander Martin\\
    1018332\\[3cm]
    Datum: \today\\[2cm]

    Prüfer: Prof.~Dr.~rer.~nat.~Christoph Quix\\
    Zweitprüfer: Sayed Hoseini,~M.Sc.

  \end{center}
\end{titlepage}
\newpage
\section*{Eidesstattliche Erklärung}

\begin{tabbing}
  Name: \hspace{4em}\= Alexander Martin\\
  Matrikelnr.: \> 1018332\\
  Titel: \> Entwurf und Implementierung einer generischen \\ 
         \> Ingestion-Schnittstelle mit Versionierung für Data-Lake-Systeme
\end{tabbing}

Ich versichere durch meine Unterschrift, dass die vorliegende
Arbeit ausschließlich von mir verfasst wurde.
Es wurden keine anderen als die von mir angegebenen Quellen und Hilfsmittel
benutzt.

Die Arbeit besteht aus \underline{\hspace{3em}} Seiten.

\vspace{8ex}
\begin{tabbing}
  \underline{\hspace{14em}} \hspace{3em}\= \underline{\hspace{14em}} \\
  Ort, Datum \> Unterschrift
\end{tabbing}
\newpage
\section*{Zusammenfassung}
asd
\section*{Abstract}
asd
\newpage

\pagestyle{fancy}
\tableofcontents
\newpage

\chapter{Einleitung}

\section{Die Motivation}
In der heutigen Zeit spielen Daten in der Welt eine immer größere Rolle.
In \textit{Rethink Data Report 2020} \textcite{rethink_data_2020} wurde eine Studie durchgeführt, die eine Steigerung von 42\% der Menge an anfallenden Daten pro Jahr prognostiziert.
Dies wird unter anderem auf den vermehrten Einsatz von IoT-Geräten, immer ausführlicheres Analysieren von Daten und die einfacher werdende Anwendung von Cloud-Speichern zurückgeführt.
Dabei besteht die Herausforderung, Daten in verschiedensten Formaten in großen Mengen zu verwalten und zu verwenden.

Ein Lösungansatz für dieses Problem sind Data-Lake-Systeme.
Data-Lake-Systeme sind zentrale Datenspeicher, die strukturierte, semi- und unstrukturierte Daten in ihrem Rohformat speichern.
Mit Hilfe von Metadaten bietet ein Datalake Schnittstellen zur Datenanalyse und -abfrage.
Dabei funktioniert das System nach dem Schema-On-Read oder auch ELT (Extrahieren Laden Transformieren) Prinzip.
Das bedeutet, dass die Daten wie bereits erwähnt im Rohformat im Data Lake gespeichert werden und erst nach dem Laden ein entsprechendes Schema angewendet wird.

Es gibt bereits viele Anbieter, die fertige Data-Lake-Systeme anbieten.
Dabei ist jedoch ein Nachteil, dass sie häufig nur in der (Cloud-)Infrastruktur des Anbieters (z.B. Microsoft Azure\footnote{https://azure.microsoft.com/}, Amazon Web Services\footnote{https://aws.amazon.com/}) verfügbar sind und sich ihrere Architektur nach diesen Diensten richtet.
Daher wird in dieser Arbeit eine Schnittstelle für die Ingestion, also das Laden der Daten in das Data-Lake-System, entwickelt, die in einem platform unabhängigen Data-Lake-System Anwendung finden soll.

\section{Der Aufbau}
Am Anfang wird auf die Ziele eigengangen, die Schnittstelle erreichen soll.
Aus diesen Zielen werden dann konkrete Anforderungen an die Entwicklung abgeleitet.
Im zweiten Kapitel wird das System der Schnittstelle entwickelt, ohne dabei auf konkrete Details, wie zum Beispiel Programmiersprachen, einzugehen.
Hier geht es mehr um die Architektur und das Design, das benötigt wird um alle Anforderunegn abzudecken.
Danach wird die Umsetzung beschrieben.
Hierbei spielt vorallem das bereits exisitierende Data-Lake-System, in dem die Ingestion-Schnittstelle integriert werden soll, eine große Rolle.
Zum Schluss wird die Ingestion-Schnittstelle evaluiert und ein Ausblick auf mögliche weitere Arbeiten gegeben.

\section{Das Existierende System}
In dem Masterprojekt \textit{Development of a Data Lake System} \parencite{datalake_proj} an der Hochschule Niederrhein wurde bereits eine Data-Lake-System-Prototyp entwickelt.
Das System ist eine monolithische Client-Server-Anwendung.
Es besteht aus einer REST-API, die zur Interaktion mit dem Data-Lake-System verwendet wird und einem Web-Frontend.
Außerdem können durch einfach Anpassungen der Server-Anwendung beliebige Datenspeicher in das Data-Lake-System integriert werden.

Als Basis wird \textit{Apache Spark}\footnote{https://spark.apache.org/} verwendet.
\textit{Apache Spark} ist eine Plattform, um Analyse auf großen Datenmengen aus zu führen.
Außerdem gibt es Schnittstellen für \textit{Scala, Java und Python} und es wird auch die Verarbeitung von Datenströmen und maschinelles Lernen unterstützt \parencite{spark}.

Der Server ist in \textit{Python} geschrieben und verwendet das Framework \textit{Flask}\footnote{https://palletsprojects.com/p/flask/} um eine REST-API bereitzustellen, über die mit dem Data Lake interagiert werden kann.
Dabei werden über die API \textit{JSON}-Objekte ausgetauscht, so dass die API client-unabhängig verwendet werden kann.
Der Client des Projekts ist eine Webanwendung, die mit \textit{Angular}\footnote{https://angular.io/} umgesetzt wurde.

\section{Verwandte Arbeiten}

\subsection{Techniken}

\subsubsection{Apache Spark}

\subsubsection{Apache Kafka}

\textit{Apache Kafka} ist ein verteiltes Event-Streaming-System, dass nach dem Publish-Subscribe-Muster funktioniert.
Events können von Produzenten in das System veröffentlicht werden und Konsumenten können diese Events abonnieren.
Das ganze läuft dabei in Echtzeit ab.
Durch seine Verteilung kann \textit{Kafka} den Ausfall einzelner Server ausgleichen.
Außerdem können Ströme von Events für einen beliebigen Zeitraum abgespeichert werden.

\textit{Kafka} besteht aus einem Cluster von Servern und verschiedenen Clients.
Es gibt zwei Arten von Servern.
Einige bilde die Speicherebene von \textit{Kafka} und werden Broaker genannt.
Die anderen verwenden \textbf{Kafka Connect}\footnote{https://kafka.apache.org/documentation/\#connect} um existierende Systeme, zum Beispiel eine Datenbank, in das Kafka Cluster zu integrieren.
Anwendungen, die entweder Events produzieren oder konsumieren sind die Clients.

In diesem System repräsentiert ein Event den Fakt, dass etwas "`passiert"' ist und besteht aus einem Schlüssel, einem Wert, einem Zeitstempel und optionalen Metadaten.
Dabei werden die Werte nicht interpretiert sonder einfach als Block versendet und können so beliebige Struktur haben.
Events werden in sogenannte Topics unterteilt.
Es kann immer mehrere Produzenten oder Konsumenten auf einer Topic geben.
Events in einer Topic können mehrfach gelesen werden und werden nicht nach dem Konsumieren gelöscht.
Es kann aber für jede Topic einzeln eine Dauer festgelegt werden, nach der die Events verworfen werden.
Um eine Topic fehlertolerant zu machen, kann diese repliziert werden.

Topics werden in Partitionen über verschiedene Broaker aufgeteilt, so dass das ganze System gut skalierbar wird.
Ein Produzenten kann zum Beispiel Events auf mehreren Brokern gleichzeitig veröffentlichen.
Wenn ein Event in einer Topic veröffentlicht wird, wird dieses an eine der Partitionen angehängt.
Events, die den gleichen Schlüssel haben werden immer der gleichen Partition zugeordnet und Events einer Partition kommen garantiert in der Reihenfolge des Schreibens bei dem Konsumenten der Partition an \parencite{kafka-docs}.

\textit{Apache Kafka} wird im Big Data Bereich weit verbreitet um Datenströme zu verarbeiten.
Daher macht es Sinn, \textit{Kafka} auch in dieses Data Lake System zu integrieren und darin bereit zu stellen.
Außerdem kann es auch für die Kommunikation zwischen den verschiedenen Microservices verwendet werden.

\subsection{Hadoop Distributed File System}

Als Speicher wird das \textit{Hadoop Distributed File System (HDFS)} verwendet.
Das \textit{HDFS} ist ein verteiltes, auf große Dateien ausgelegtes Dateisystem.
Ein \textit{HDFS} Cluster besteht aus einem Namenode und mehreren Datanodes.

Der Namenode verwaltet den Baum des Dateisystems und kontrolliert den Zugriff durch Clients.
Zusätzlich führt er Operation auf dem Dateisystem aus.
Dazu zählen das Öffnen, Schließen oder Umbenennen von Dateien oder Ordnern.
Die Dateien selbst werden in Blöcke aufgeteilt auf den Datanodes gespeichert.
Diese sind auch dafür verantwortlich, Lese- und Schreibanfragen zu bedienen und verwalten die Erstellung, Löschung und Replikation unter Anleitung des Namenodes \parencite{hdfs}.

Das \textit{HDFS} ist ebenfalls eine weit verbreitete Technik im Big Data Bereich und eignet sich auch hier, durch die Auslegung auf große Dateien, sehr gut um die Quelldateien der Datenquellen abzulegen.
Außerdem sind Dateien auf allen Servern verfügbar, da es sowohl eine REST-Schnittstelle als auch Unterstützung in \textit{Apache Spark} gibt.

\chapter{Anforderungen}
\label{sec:anf}

Basierend auf der Zielsetzung (\cref{sec:ziel}) können die Anforderungen erarbeitet werden, nach denen die Ingestion-Schnittstelle entwickelt werden soll.
Dafür werden nachfolgend die Ziele in einzelne Abschnitte aufgeteilt.
In diesen Abschnitt wird das Ziel nochmal genauer erläutert und die einzelnen Anforderungen herausgearbeitet.
Auf die Erfüllung der Ziele und Anforderungen wird dann nochmal in der Evaluierung und Zusammenfassung eingegangen.

\section{Quellen- und Formatunabhänigkeit}
\label{sec:anf-unab}
Am HIT werden verschiedenste Daten verarbeitet.
Viele kommen aus Datenbanksystemen oder Dateien.
Es gibt aber auch spezielle Systeme, die ihre Daten nur über eine REST-API zur Verfügung stellen.
Alle diese Daten sollen mit der Ingestion-Schnittstelle geladen und gespeichert werden können.
Dazu muss diese sowohl Daten entgegen nehmen als auch aus Systemen extrahieren können und im Data-Lake-System muss ein Speicher integriert sein, der Daten aus allen Strukturen und Formaten speichern kann.
Um eine Flexibilität auch beim Einsatz des Systems in einer anderen Umgebung zu gewährleisten soll nicht nur der interne, sondern auch externe Systeme zum Speichern der Daten verwendet werden können.

\paragraph{ANF\_01}
\label{ANF_01}
Die Schnittstelle muss in der Lage sein Quelldaten entgegen zu nehmen, die an das Data-Lake-System gesendet werden.
Diese müssen so verwaltet werden, dass sie über Apache Spark gelesen werden können.

\paragraph{ANF\_02}
\label{ANF_02}
Da Apache Spark nicht von sich aus in der Lage ist, alle Datenformate zu verstehen, muss es möglich sein, die SparkSession mit benötigten Paketen zu erweitern.

\paragraph{ANF\_03}
\label{ANF_03}
Für die Unterstützung verschiedenster Quell- und Zielspeicher verwendet Apache Spark zum Lesen und Speichern einen Format-Parameter und variable Optionen.
Diese sollen komplett konfigurierbar sein um alle Systeme verwenden zu können.

\paragraph{ANF\_04}
\label{ANF_04}
Einige Quellen, zum Beispiel eine REST-API können nicht direkt über Spark in ein DataFrame gelesen werden.
Daher muss eine Möglichkeit geben werden, die Ingestion zur Laufzeit um eigenen Programmcode zu erweitern, der ein DataFrame manuell aus einer Reihe von Abfragen aufbaut.

\section{Kontinuierliches Laden}
\label{sec:anf-ci}
Da Daten sich mit der Zeit ändern, soll die Ingestion-Schnittstelle in der Lage sein, neue Daten aus einer Datenquelle, die bereits aufgenommen wurde, erneut zu laden.
Die Implementierung soll das erneute Anstoßen, eine Zeitsteuerung oder Datenströme zulassen.

\paragraph{ANF\_05}
\label{ANF_05}
Es soll möglich sein, Datenströme in das Data-Lake-System zu integrieren und als Quelle für kontinuierliche Daten zu verwenden.

\paragraph{ANF\_06}
\label{ANF_06}
Um aktuelle Daten aus Datenquellen, die nicht über einen Datenstrom verfügen, zu integrieren, soll es eine  wiederholte Ausführung mit einer Zeitsteuerung geben.

\paragraph{ANF\_07}
\label{ANF_07}
Es wird ein API-Endpunkt benötigt, über den die Ingestion für eine bestimmte Datenquelle angestoßen werden kann.
Dieser soll auch dazu verwendet werden, externe Systeme, wie eigene CDC-Lösungen anzubinden.

\paragraph{ANF\_08}
\label{ANF_08}
Eine gleichzeitige Ausführung mehrerer Ingestions auf der gleichen Datenquelle könnte leicht zu Konflikten in den Daten führen.
Daher soll sichergestellt werden, dass das System diesen Fall nicht zulässt.


Wie bereits erwähnt, ist der Prototyp des Data-Lake-Systems eine monolithische Anwendung.
Das bedeutet, die gesamte Anwendung ist als Komplettlösung in einem Programm entwickelt und bereitgestellt worden.
Solche Ansätze sind Anfangs leichter umzusetzen, haben aber größere Nachteile in Bereichen wie Fehlertoleranz und Wartbarkeit.
Daher soll für die Ingestion-Schnittstelle der Microservice-Ansatz verfolgt werden.
Hierbei werden die Funktionalitäten und Aufgaben auf mehrere kleinere Anwendungen aufgeteilt.
Das hat, wie von \textcite{microservices} dargestellt, mehrere Vorteile.
Die Wartung fällt bei mehreren kleinen Programmen leichter, da sie übersichtlicher und verständlicher sind.
Bei Fehlfunktionen einzelner Mircoservices fällt außerdem nicht die komplette Anwendung aus, sondern nur die Funktion, für die der Service zuständig war.
Zuletzt ist es einfacher bestimmte Aspekte der Software zu skalieren und bei Updates bleibt eine höhere Verfügbarkeit, da nur ein kleiner Teil des Systems neu gestartet werden muss.

\section{Datenversionierung}
\label{sec:anf-vers}
Die Änderungen, die an Daten gemacht werden, sind wichtige Informationen, um weitere Verarbeitungen oder Analysen zu optimieren.
Gerade durch die kontinuierliche Ingestion von neuen Daten am HIT entsteht eine Vielzahl solcher Änderungen.
Dabei gibt es zwei verschiedene Arten, wie Änderungen in das Data-Lake-System eingespielt werden können.
Die erste Art ist die Ingestion von Daten aus einer externen Change-Data-Capture-Lösung (\cref{sec:cdc}) als Änderungen an einer Datenquelle, die bereits ins System geladen wurde.
Bei der zweiten Art wird eine Datenquelle, die bereits geladen wurde, ein weiteres mal geladen.
Hierbei ist es notwendig, das die Ingestion-Schnittstelle die Änderungen zwischen den beiden Ständen erkennen kann.
Die Änderungen an den Daten im System sollen über eine Versionierung nachvollziehbar gespeichert werden.

\paragraph{ANF\_09}
\label{ANF_09}
Das System soll eine Möglichkeit bieten, Änderungen an Daten als Versionen zu speichern und zur Abfrage zur Verfügung zu stellen.
Außerdem sollen damit auch die Daten zu bestimmten Zeitpunkten rekonstruierbar sein.

\paragraph{ANF\_10}
\label{ANF_10}
Um für alle eingehenden Daten die Möglichkeit der Versionierung im System anbieten zu können, muss eine CDC-Implementierung eingebaut werden, die für jede Datenquelle ausgeführt werden kann.

\paragraph{ANF\_11}
\label{ANF_11}
Auch die Verwendung einer eigenen CDC-Lösung für eine Datenquelle soll unterstützt werden.
Dazu muss eine Quelle mit Änderungsdaten für eine bereits aufgenommen Datenquelle erstellt werden können.

\section{Datenversionierung}
\label{sec:anf-metadata}
Metadaten spielen eine wichtige Rolle bei der Qualität der geladen Daten.
Die Ingestion-Ebene kann schon beim Laden der Daten Metadaten erfassen.
Für diese wird ein Metadatenmodell benötigt, dass alle Metadaten abbildet, die bei der Ingestion erfasst werden können.

\paragraph{ANF\_12}
\label{ANF_12}
Die Ingestion soll beim Laden der Daten Metadaten erfassen.

\paragraph{ANF\_13}
\label{ANF_13}
Das Metadatenmodell soll alle Informationen enthalten, die bei der Ingestion gesammelt werden können.

\section{Architektur}
\label{sec:anf-arch}
Die Architektur soll die Ingestion in Mircoservices umgesetzt werden.
Außerdem wird eine Schnittstelle für die Interaktion mit dem System benötigt.
Daraus ergeben sich folgende Anforderungen, an die Architektur.

\paragraph{ANF\_14}
\label{ANF_14}
Die Interaktionsschnittstelle mit dem System soll eine REST-API sein, die in das aktuelle Prototyp-System integriert werden soll.

\paragraph{ANF\_15}
\label{ANF_15}
Durch eine klare Trennung der Aufgaben der Microservices soll deren Überschneidungen und Abhängigkeiten so gering wie möglich gehalten werden.

\paragraph{ANF\_16}
\label{ANF_16}
Für die Kommunikation zwischen den Microservices soll eine einheitliche Lösung verwendet werden.
Diese soll es auch ermöglichen neue Mircoservices einfach in die Architektur einzubringen.


\chapter{Entwurf}

Der Ingestion-Prozess ist als erster Schritt im Lebenszyklus der Daten maßgebend für deren Qualität und Aussagekraft bei der späteren Verarbeitung und Analyse \parencite{ingestion_01}.
Daher muss schon bei dem Entwurf nicht nur auf die Anforderungen Rücksicht genommen werden, sondern auch auf das fertige Data-Lake-System.

\section{Architektur}
\label{sec:arch}

Zu Anfang ist es sinnvoll, die Architektur des Systems festzulegen.
Diese bestimmt, welche Komponenten und Microservices entwickelt und verbunden werden müssen.
Der erste Schritt dabei ist es, die Aufgaben aufzuteilen, die das System erfüllen soll.
Diese Aufgaben werden dann auf Microservices verteilt.
Die weitere Entwicklung des Systems stützt sich dann auf die fertige Architektur.

\subsection{Aufgabenverteilung}
Durch die Verwendung von \textit{Apache Spark} ist es nicht sinnvoll, dass Laden der Daten, die Deltaerkennung und das Speichern zu trennen.
In \textit{Apache Spark} werden alle drei Arbeitsschritte auf einem DataFrame ausgeführt.
Das heißt, dass dieses zwischen den Microservices ausgetauscht werden müsste, was zu einem erheblichen Aufwand führen würde.
Aus diesem Grund kann die logische Aufteilung in Ingestion, Deltaerkennung und Speicherung nicht auch als Aufgabenverteilung verwendet werden.

Besser trennbare Aufgaben findet man bei der Betrachtung der technischen Seite.
Hierbei gibt es die API, die Ingestion in \textit{Apache Spark} und das kontinuierliche Ausführen.

Bei der \textbf{API} handelt es sich um den Service für die Interaktion mit dem Data-Lake-System.
Durch die Anforderungen ist bereits festgelegt, dass dieser ein Web-Server mit einer REST-Schnittstelle ist.
Es geht zwar in dieser Arbeit nur um die Ingestion, aber der API-Service sollte Schnittstellen zu allen Funktionen des Data-Lake-Systems enthalten.

Die \textbf{Ingestion} ist dafür zuständig, die Datenquellen zu verarbeiten und den kompletten Prozess vom Laden bis zum Speichern der Daten in \textit{Apache Spark} auszuführen.
Die Ausführung soll für eine Datenquelle nur einmal gleichzeitig aber parallel für unterschiedliche laufen.

Bei einer zeitgesteuerten oder Datenstrom-Ingestion muss die \textbf{kontinuierliche Ausführung} sichergestellt werden.
Für alle Datenquellen muss regelmäßig geprüft werden, ob für diese gerade eine Ingestion ausgeführt wird und ausgeführt werden sollten.
Falls keine ausgeführt wird aber sollte, wird die Ingestion für diese Datenquelle gestartet.

\subsection{Komponenten und Microservices}
Die drei oben genannten Aufgaben können jeweils einem Microservice zugeordnet werden.
In \fref{fig:ingestion_arch} ist eine dazu passende Architektur zu sehen.
Der API-Service übernimmt die REST-Schnittstelle, der Ingestion-Service kümmert sich um die Ausführung der Ingestion und der Continuation-Service stellt die kontinuierliche Ausführung sicher.

Neben diesen Mircoservices wird noch ein Nachrichtensystem benötigt.
Das Nachrichtensystem stellt die Kommunikation zwischen den Microservices dar.
Hier ist es wichtig, dass es einem Sender möglich ist, Nachrichten an einen oder auch an mehrere Empfänger zu senden.
So soll sichergestellt werden, dass bestehende Microservices einfach repliziert und neue eingefügt werden können.

Zum Ablegen von internen Informationen wird eine Datenbank benötigt.
Alle Microservices haben zugriff auf diese Datenbank und können Daten in ihr bearbeiten.
Es handelt sich dabei aber nicht um den internen Speicher für geladene Daten des Data-Lake-Systems sondern nur um Daten, die für den Betrieb des Systems benötigt werden.
Darunter fallen zum Beispiel Authentifizierungsdaten oder Verbindungsinformationen von Datenquellen.

\begin{figure}
  \centering
  \includegraphics{Grafiken/ingestion-arch.pdf}
  \caption{Architektur der Ingestion Komponenten}
  \label{fig:ingestion_arch}
\end{figure}

\section{Plugins}

In \nameref{ANF_04} wird gefordert, dass zusätzlicher Code bei der Ingestion ausgeführt werden können soll.
Das kann durch Plugins umgesetzt werden.
Plugins können einer Datenquelle hinzugefügt und an verschiedenen, fest definierten Punkten ausgeführt werden.
Da die Plugins eventuell auf Software-Bibliotheken zurückgreifen müssen, die nicht auf dem Data-Lake-System vorhanden sind, kann zusätzlich eine Liste von Abhängigkeiten der Plugins angegeben werden.
Das Prinzip der Plugins kann beim Ausbau auch über die Ingestion hinaus im System angewendet werden.

Für die Ingestion gibt es zwei Stellen, an denen die Möglichkeit für Plugins gegeben sein sollte.
Die erste Möglichkeit ist, wie durch die Anforderung gefordert, das Laden von Daten.
Genauer bedeutet das, dass das Plugin die Aufgabe übernimmt das DataFrame zu erstellen, welches später wieder gespeichert wird.
Das Plugin ersetzt die normale Funktion zum Laden in ein DataFrame.
Dies wird zum Beispiel bei der Ingestion von Daten aus einer REST-API benötigt.
Hier muss ein DataFrame manuell aus Daten erzeugt werden, die erst über einen REST-Client abgefragt werden.

Als zweite Möglichkeit sollte es Plugins geben, mit denen man nach dem Laden den DataFrame manipulieren kann.
Streng genommen widerspricht diese Möglichkeit dem Data-Lake-Prinzip, Daten unverändert in ihrem Rohzustand zu speichern.
Allerdings ist zum Beispiel bei der Anbindung von Kafka-Datenströmen eine solche Nachbearbeitung sinnvoll, da die Daten nur als Byte-Block übertragen werden.
Für solche Fälle sollte es möglich sein, die unstrukturierten Byte-Daten in ein strukturiertes Format zu überführen.
\input{Kapitel/Entwicklung/Metadaten}
\section{API-Service}

Der API-Service benötigt nicht viel Entwicklung.
Es müssen lediglich die Enspunkte definiert werden, die Zugriff auf die verschiedenen Funktionen der Ingestion-Schnittstelle bereitstellen.
Dazu gehören die Verwaltung von Datenquellen und das Starten einer Ingestion.
Da es sich um eine REST-Schnittstelle handelt, werden die Endpunkte hier durch einen Pfad und eine HTTP-Methode definiert (\fref{tab:endpunkte}).
Außerdem kümmert sich der API-Service um die Erstellung von DatasourceDefinitions und deren Revisionen und IngestionEvents.
Bei Anfragen zum Starten einer Ingestion versendet der API-Server eine Nachricht mit der Id einer Definition der zu ladenen Datenquelle.

    {\renewcommand{\arraystretch}{1.8}
        \begin{table}[ht]
            \centering
            \begin{tabularx}{\linewidth}{|lX|}
                \hline
                GET  & /datasources                                                                         \\
                \multicolumn{2}{|l|}{Liefert alle im System gespeicherten Datenquellen}                     \\
                \hline
                GET  & /datasources/\textless id\textgreater                                                \\
                \multicolumn{2}{|l|}{Liefert die Datenquelle mit der im Pfad übergebenen Id}                \\
                \hline
                POST & /datasources                                                                         \\
                \multicolumn{2}{|l|}{Bearbeitet die Daten Datenquelle mit der im Pfad übergebenen Id}       \\
                \hline
                PUT  & /datasources/\textless id\textgreater                                                \\
                \multicolumn{2}{|l|}{Erstellt eine neue Datenquelle}                                        \\
                \hline
                GET  & /datasources/\textless id\textgreater/run                                            \\
                \multicolumn{2}{|l|}{Startet eine Ingestion der Datenquelle mit der im Pfad übergebenen Id} \\
                \hline
            \end{tabularx}
            \caption{Endpunkte des API-Servers}
            \label{tab:endpunkte}
        \end{table}
    }

\section{Continuation-Service}

Für die Sicherstellung der korrekten Ausführung kontinuierlicher Ingestions, müssen regelmäßig alle Datenquellen überprüft werden.
Dabei gibt es zwei Bedingungen nach denen entscheiden wird, ob eine Ingestion ausgeführt werden muss.
Bei Datenströmen gilt allgemein, wenn dieser nicht läuft, dann muss die Ingestion automatisch neu gestartet werden.
Die einzige Ausnahme ist, wenn die Ingestion durch den Benutzer explizit beendet wurde.

Der zweite Fall ist die Zeitsteuerung.
Für eine zeitgesteuerte kontinuierliche Ingestion werden einer Datenquelle ein oder mehrere Timer hinzugefügt.
Der Continuation-Service kontrolliert, ob der Timer zum Zeitpunkt, an dem die Datenquelle überprüft wird zutrifft oder nicht.
Wenn das der Fall ist und bisher keine Ingestion auf der Datenquelle läuft, dann wird eine neue Ingestion gestartet.

In Unix-Systemen gibt es bereits eine Lösung für die Notation solcher Timer.
Dort gibt es die sogenannten Cron-Jobs, mit deren Hilfe Aufgaben automatisch und regelmäßig ausgeführt werden können.
Dabei wird der Zeitpunkt der Ausführung über fünf Felder festgelegt.
Diese geben die Minute, die Stunde, den Tag des Monats, den Monat und den Tag der Woche als Zahlen an.
Als Erweiterung kann man "`*"' als Platzhalter für alle möglichen Werte verwenden, man kann mehrere Werte als mit Kommata getrennt angeben oder mit "`/x"' eine Liste in Schritten der Größe $x$ erzeugen.

Diese Notation soll auch für die Zeitsteuerung der Ingestions genutzt werden.
Als Referenz wird dabei die koordinierte Weltzeit (UTC) genommen, damit die Ausführung unabhängig vom Standort einheitlich bleibt.
Wenn eine Datenquelle mehrere Timer hat, reicht es aus, dass einer von diesen zutrifft.
\section{Ingestion-Service}
\label{sec:entw-ingestion}

Der Ingestion-Service hat die Aufgabe eine Ingestion für eine Datenquelle durchzuführen.
Dazu gehören das Laden der Daten in ein DataFrame, die Deltaberechnung und das Speichern.
Das meiste davon wird jedoch nicht von dem Service selbst, sondern auf dem Spark Cluster gemacht.
Der Service führt die Vorbereitung und das Deplyoment des Jobs auf dem Cluster aus.

Der Ingestion-Service wartet auf die Nachricht zur Ausführung einer Ingestion, mit der Id der DatasourceDefinition.
Als erstes wird dann geprüft, ob bereits eine Ingestion der Datenquelle aktiv ist.
Falls das nicht der Fall ist, wird ein neuer Prozess gestartet, indem die Ingestion ausgeführt wird.
Auf diese Art können mehrer Ingestion von verschiedenen Quellen parallel bearbeitet werden.

Der Ablauf einer einzelnen Ingestion kann unabhängig vom Lese- und Schreib-Typ in einem allgemeinen Ablauf abgebildet werden.
Als erstes wird die Ingestion vorbereitet.
Hier werden die Plugins installiert und eine SparkSession erstellt.
Im nächsten Schritt werden die Daten aus der Quelle geladen.
Wenn es sich dabei um Ändeurngsdaten aus einer Updatequelle handelt, können diese direkt in den entsprechenden Zieldatensatz eingepflegt werden.
Ist das nicht der Fall, foglt eine Entscheidung, ob Änderungsdaten berechnet werden müssen.
Es gelten die folgenden zwei Regeln: \begin{itemize}
    \item der Speicher-Typ ist Delta
    \item es ist nicht die erste Ingestion dieser Datenquelle
\end{itemize}
Wurden Änderungsdaten berechnet, werden diese eingepflegt und ansonsten einfach gespeichert.
Handelt es sich nicht um einen benutzerdefinierten Speicher, werden die alten Daten mit den neuen überschrieben.

\begin{figure}
    \centering
    \includegraphics[width=\textwidth]{Grafiken/Entwicklung-Ingestion-Ablauf.pdf}
    \caption{Ablauf einer Ingestion}
    \label{fig:ingestion-ablauf}
\end{figure}
\chapter{Umsetzung}

Für eine Umsetzung der entwickelten Architektur ist zuerst die Frage der Techniken zu klären.
Dazu zählen die verwendete Programmiersprache, Frameworks und fertige Software.

\section{Programmiersprache}

Durch die Verwendung von Apache Spark ist die Auswahl der Programmiersprachen auf Java, Python und Scala eingegrenzt.
Da der Prototyp in Python geschrieben wurde und dieser zum Api-Service erweitert werden soll, wird hierfür Python verwendet.
Auch für die Implementierung des Ingestion-Service bietet Python einige Vorteile.

Bei der Verwendung von Java und Scala wird ein fertig kompilierter Job mit dem Befehl "`spark-submit"' zur Ausführung an Spark gesendet.
Das bedeutet, dass der Code dieser Jobs bereits feststehen muss.
Neben dieser Option gibt es in Python auch die Möglichkeit, dass der Interpreter zur Laufzeit um die korrekte Ausführung der Jobs kümmert.
So ist es möglich für jede Anfrage speziell konfigurierte Jobs zu erstellen, die nicht vorher schon kompiliert werden und somit feststehen müssen.
Das senkt die Komplexität bei der Entwicklung der Ingestion \parencite{pyspark-int}.

Auch für die Plugins hat Python einen Vorteil.
Man kann dynamisch Programmcode aus Dateien laden und inspizieren.
So können für jede Ausführung die Plugins einer Datenquelle frisch geladen werden.
Es muss nur dafür gesorgt werden, dass auch alle Abhängigkeiten erfüllt sind.

Um die Entwicklung einheitlich zu halten, wird auch der Continuation-Service in Python implementiert.
\section{Nachrichtensystem}

Für die Übermittlung von Nachrichten zwischen verschiedenen Anwendungen gibt es sogenannte Message-Broker.
Diese koordinieren als Mittelsmann die Verteilung der Nachrichten an verschiedene Empfänger.
Das hat den Vorteil, dass der Sender unabhängig von den Empfängern wird und die Kommunikation asynchron statt finden kann \parencite{message-broker}.

Es gibt mittlerweile einige Projekte, diese Aufgabe auf verschiedene Arten lösen.
Hier wird dafür \textit{Apache Kafka} verwendet, welches im Big Data Bereich weit verbreitet ist um Datenströme zu verarbeiten.
Daher macht es Sinn, dieses in das Data-Lake-System zu integrieren und darin bereit zu stellen.
Um das System dabei nicht unnötig komplex und zu groß werden zu lassen, wird daher auf einen anderen Message-Broker verzichtet.

Durch die Verwendung von \textit{Kafka} als Message-Broker müssen die Topics der Nachrichten festgelegt werden.
Hier kann es zu Konflikten sowohl mit internen Nachrichten als auch bei anderen Datenströmen kommen, wenn die Namen falsch gewählt werden.
Daher sollten Topics, die zur internen Kommunikation des Data-Lake-Systems gehören immer mit "`dls\_\_"' als Prefix benannt werden.
Danach folgt der Bereich in dem die Nachricht verwendet wird, hier zum Beispiel "`ingestion"'.
An diesen Namen kann dann noch weiter Unterscheidung angehängt werden.
Im Fall der Nachricht für das Ausführen einer Ingestion wäre dann die Topic "`dls\_\_ingestion\_\_run"'.

Wenn mehrere Consumer in einer Gruppe für eine Topic sind, werden Nachrichten nicht an alle sonder immer nur an einen aus der Gruppe gesendet.
Dieser Mechanismus kann für den Lastausgleich an bestimmten Stellen verwendet werden.
Für die Ingestion kann so der Ingestion-Services einfach repliziert werden.
\section{Datenbank und Datenmodell}

Da das Datenmodell einer Datenquelle Listen von anderen Datenmodell enthält, bietet sich hier als einfachste Lösung die Verwendung einer dokument-orientierten NoSQl-Datenbank an.
Das hat den Vorteil, dass diese Listen direkt in den Objekten der Datenquellen abgelegt werden können.
In Datenbanken, die Tabellen verwenden, müsste man für jedes Modell eine eigene Tabelle erstellen und die Verknüpfungen über über JOIN-Operationen auflösen.
Bei jeder Abfrage einer Datenquelle aber auch die verknüpften Einträge der Ingestion-Events oder Revisionen gebraucht werden.
Außerdem gibt es viele Anfragen auf die Datenquellen, da diese nicht zwischen den Mircoservices ausgetauscht werden und so nicht im Speicher vom Service verwaltet werden können.
Daher ist es effizienter die relevanten Daten direkt mit einer Abfrage laden zu können.
Das Speichern der Felder für die Optionen beim Lesen und Schreiben wird auch dadurch erleichter, dass es kein Schema gibt.
Man kann so die Optionen einfach in einem Dokument in der Revision speichern.

Hier kommt \textit{MongoDB}\footnote{https://www.mongodb.com/} als Datenbank zum Einsatz.
\textit{MongoDB} kann frei verwendet werden und bei bei größeren Datenmengen verteilt eingesetzt werden.
\fref{fig:datamodel} zeigt einen Überblick, wie das entwickelte Datenmodell aus \ref{sec:datasourcemodel} gespeichert wird.

\section{Interner Datenspeicher}

Die Entwicklung einer eigenen Lösung zum Speichern von Daten mit Versionierung passt nicht in den zeitlichen Rahmen dieser Arbeit.
Daher wird der Delta Lake verwendet, da dieser alle benötigten Funktionen bietet und eine volle Schnittstelle zu Spark hat.
Da das System auf eigenen Server laufen soll, wird als Speicher für den Delta Lake ein HDFS Cluster verwendet.
Außerdem können auch die Quelldateien der Ingestion und die Plugins im HDFS abgelegt werden und so zentral durch alle Microservices erreicht werden.
Der Zugriff zum Speichern der Daten geschieht über den Delta Lake beziehungsweise Spark und für die Quelldateien und Plugins über die WebHDFS REST-Schnittstelle.

Wenn eine Datenquelle entweder mit Versionierung gespeichert wird oder Quelldateien oder Pluigns enthält, wird für diese ein Ornder im HDFS angelegt.
Diese enthält dann einen Unterordner für die Delta Tabelle und für jede Revision.
In den Revisionsordnern werden dann zugehörige Quelldatein und Plugins abgelegt. 
\section{Api-Service}

Der Api-Service basiert auf dem Server aus dem Masterprojekt, ein mit Flask\footnote{https://flask.palletsprojects.com/} in Python geschriebener Web-Server.
Um eine saubere Trennung zwischen den existierenden und den neuen REST Endpunkten zu erreichen, wird die alte API unter den Pfad "`/api/v1/"' verschoben und die neue unter "`/api/v2/"' erstellt.
Kommunikation mit dem Data Lake System findet nur statt, wenn eine Anfrage zur Ausführung einer Ingestion bearbeitet wird.
Hier wird ein Event mit dem Schlüssel "`dls\_\_ingestion\_\_run"' erzeugt.
\section{Continuation-Service}

Der Continuation-Service ist ein einzelner Prozess.
In einer Schleife werden erst alle Datenquellen überprüft, die eine Zeitsteuerung enthalten.
Hier wird eine Ingestion gestartet, wenn der Status des letzten IngestionEvents "`STOPPED"' oder "`FINISHED"' ist.
Im zweiten Schritt werden alle Datenquellen mit Datenströmen geprüft.
Datenströme werden nur bei einem Status von "`FINISHED"' automatisch gestartet, da "`STOPPED"' bedeutet, dass die Ausführung beabsichtigt beendet wurde.
Dabei wird die Schleife maximal einmal jede Minute durchlaufen, da über die Zeitsteuerung keine schneller Ausführung möglich ist.
In \cref{algo:ci-loop} ist die Logik eines Durchlaufes zu sehen.

\begin{algorithm}
    \caption{Continuation loop}
    \label{algo:ci-loop}

    loopStart $\gets$ now \

    definitions $\gets$ query DatasourceDefinitions with continuation timers \

    \ForAll {definition in definitions} {

        timers $\gets$ timers of source \

        \If {state is STOPPED or FINISHED} {

            \ForAll {timer in timers} {

                \If {timer is now} {

                    send run event \

                    end check for this DatasourceDefinition \
                }
            }
        }
    }

    definitions $\gets$ query DatasourceDefinitions with type STREAM \
    \ForAll {definition in definitions} {

        \If {state is FINISHED} {

            send run event \
        }
    }

    looEnd $\gets$ now \

    loopDuration $\gets$ loopEnd - loopStart \

    \If {loopDuration $<$ 60 sec} {

        sleep for 60 sec - loopDuration \
    }

\end{algorithm}


\section{Ingestion-Service}

Wie in \cref{sec:entw-ingestion} beschrieben, wird für eine Ingestion ein Prozess gestartet.
Zum Starten einer Ingestion wird das Topic "`dls\_\_ingestion\_\_run"' verwendet.
In \cref{sec:ingestion-run} wird der detailliere Ablauf mit den verschiedenen Wegen für Read- und SaveTypes beschrieben.
Vorher werden die benötigten Grundlagen zur Ausführung erläutert.

\subsection{Berechnen und Speichern der Änderungsdaten}
Um Änderungen in eine Delta-Tabelle zu übernehmen, müssen die Änderungsdaten in einem Format sein, dass Aufschluss darüber gibt, welche Daten hinzugefügt, welche geändert und welche gelöscht worden sind.
Das hier verwendete Format muss genau dem Schema der original Daten entsprechen.
Zusätzlich soll eine Spalte oder ein Feld auf der obersten Ebene mit dem Namen "`cd\_deleted"' vorhanden sein.
Darin ist ein Boolean-Wert enthalten, der aussagt, ob der Eintrag aus den Daten gelöscht wurde oder nicht.
Der Datensatz mit den Änderungsdaten darf außerdem nur geänderte Daten enthalten.
Aus diesem Format können, wie später gezeigt, die drei Operationen abgeleitet werden.

Um nicht auf ein externes Change-Data-Capture-System angewiesen zu sein, gibt es eine interne Lösung, die auf alle (semi-)strukturierten Daten angewendet werden kann.
Dazu eignet sich der in \cref{sec:snaps} genannte snapshot-basierte Ansatz.
Dieser ist zwar langsamer als alle anderen, aber auch als einziger unabhängig von der Datenquellen und innerhalb des Data-Lake-Systems anwendbar.

\textcite{snapshot_algos} haben einen Algorithmus vorgestellt, der diese Aufgabe mit der Hilfe von Join-Operationen löst.
Für den Vergleich werden für jeden Eintrag eindeutige Schlüssel benötigt.
Das Feld "`id\_column"' in der DatasourceDefintion enthält den Namen der Spalte oder des Feldes, das als Schlüssel fungieren soll.
Es können auch verschachtelte Felder verwendet werden.
Der Schlüssel kann aber nicht aus mehreren Feldern zusammengesetzt werden.
Die Namen der Felder in den einzelnen Ebenen müssen dafür mit einem "`."' getrennt werden.
Beispielsweise "`user.id"' für: \begin{verbatim}
    {
        user: { 
            id: x 
        }
        ...
    }
\end{verbatim}

Join-Algorithmen sind in der Informatik bereits vielfach besprochen worden.
Wenn man die Einträge beider Snapshots über ihren Schlüssel verknüpft, kann man so durch einen Vergleich der Ergebnisse alle geänderten Einträge finden.
In einem Full-Outer-Join sind zusätzlich alle Einträge vorhanden, die nur in einem der beiden Einträge vorhanden sind.
Die restlich Felder bekommen einen Null-Wert.
Je nachdem auf welcher Seite die Null-Werte stehen, handelt es sich um eine Einfügung oder Löschung.
Da durch SparkSQL eine gute Unterstützung für Join-Operationen gegeben ist, ist dieser Ansatz auch gut für den Einsatz im Data Lake geeignet.
Dieses Vorgehen funktioniert sowohl für relationale als auch semistrukturierte Daten.
Bei dem Vergleich von semistrukturierten Daten wird jedoch nur die oberste Ebene als Spalte betrachtet.
Alle darunter verschachtelten Objekte werden als Werte dieser Spalten betrachtet.

Der \cref{algo:delta-calc} zum Vergleich von DataFrames orientiert sich an diesem Vorgehen.
Als Eingabe werden ein linkes DataFrame, mit dem aktuellen internen Stand, eine rechtes DataFrame, mit den neuen Daten und der Name des Schlüssels benötigt.
Um den Vergleich der Zeilen später einfacher zu machen, wird zuerst für jede Zeile der zu vergleichenden Datensätze ein Hashwert über alle Spalten berechnet.
Außerdem wird eine weitere Spalte mit dem Namen "`\char`~ id"' hinzugefügt, die den Wert des Schlüssels zugewiesen bekommt.
Da wie bereits erwähnt bei semistrukturierte Daten nur die oberste Ebene als Spalten betrachtet wird, ist es so erst möglich  Einträge über verschachtelte Schlüssel zu verknüpfen.
Danach wird ein Full-Outer-Join über den Daten ausgeführt.
Im nächsten Schritt werden alle Zeilen, in denen die Hashwerte gleich sind aus dem Datensatz entfernt, da diese keine Relevanz für die Änderungsdaten haben.
Zu den gefilterten Daten wird nun die Spalte "`cd\_deleted"' hinzugefügt.
Ist in den alten Daten eine Zeile vorhanden, in den neuen aber nicht mehr, so ist der Hashwert auf der rechten Seite $Null$.
In diesem Fall bekommt die Spalte "`cd\_deleted"' den Wert $true$ für alle anderen ist der Wert $false$.
Da das Ergebnis der Join-Operation alle Spalten doppelt enthält, einmal von links und einmal von rechts, müssen diese noch zusammengeführt werden.
Dazu können immer, außer bei dem Schlüssel, die Werte der rechten Seite genommen werden, da diese die neuen Werte enthält.
Der Wert für den Schlüssel wird auch von der rechten Seite übernommen, es sei denn dieser ist $Null$ (bei gelöschten Zeilen), dann muss der linke Wert verwendet werden, da der Schlüssel nicht $Null$ sein darf.
Zum Schluss müssen die zusätzlich für den Vergleich hinzugefügten Spalten wieder entfernt werden.
Das Ergebnis ist ein DataFrame mit allen Änderungen.
Es hat das gleiche Schema wie die Ursprungsdaten, aber mit der zusätzlichen Spalte "`cd\_deleted"', die Auskunft darüber gibt, ob ein Datensatz gelöscht wurde.

\begin{algorithm}
    \caption{Deltaberechnung}
    \label{algo:delta-calc}
    \textbf{Input:} $left$: DataFrame, $right$: DataFrame, $id\_column$: String  \\
    \textbf{Output:} $change\_data$: DataFrame \\
    
    \ForAll{$df$ \textbf{in} $[left, right]$}{
        $df$.add\_column($name$: "`hash"', $value$: $df$.hash\_over\_all\_rows())\;
        $df$.add\_column($name$: "`\char`~ id"', $value$: $df$.get\_value($id\_column$))\;
    }

    $change\_data \gets$ $left$.join($right$, $type$: full-outer, $column$: "`\char`~ id"')\;
    $change\_data$.remove\_all\_row($condition$: $left.hash$ equals $right.hash$)\;
    $change\_data$.add\_column($name$: "`cd\_deleted"', $value$: $false$)\;

    \ForAll{$row$ \textbf{in} $change\_data.rows$}{
        \If{$row.right.hash$ \textbf{is} $Null$}{
            $row.cd\_deleted \gets true$\;
        }
    }
    
    \ForAll{$column$ \textbf{in} $left.columns$}{
        \eIf{$column$ \textbf{is} $id\_column$}{
            \ForAll{$row$ \textbf{in} $change\_data.rows$}{
                \eIf{$row.right.hash$ \textbf{is} $Null$}{
                    $row.id\_column \gets row.left.id\_column$\;
                } {
                    $row.id\_column \gets row.right.id\_column$\;
                }
            }
        }{
            \ForAll{$row$ \textbf{in} $change\_data.rows$}{
                $row.column \gets row.right.column$\;
            }
        }
        $change\_data$.remove\_column($right.column$)\;
        $change\_data$.remove\_column($left.column$)\;
    }
    $change\_data$.remove\_column("`\char`~ id"')\;
    $change\_data$.remove\_column("`hash"')\;

\end{algorithm}

Das Einpflegen der Änderungen unterscheidet sich nach dem gewählten Speicherziel.
Für Delta-Tabellen wird die Delta Lake API verwendet.
Dazu werden die Änderungsdaten mit den aktuellen Daten über den Schlüssel zusammengeführt.
Dabei können verschiedene Fälle definiert werden.
Wenn die Ids einer Zeile gleich sind und die Spalte "`cd\_deleted"' $true$ enthält, wird die Zeile aus den Daten gelöscht, ansonsten wird der Datensatz aktualisiert.
Wenn die Ids nicht übereinstimmen und die Spalte "`cd\_deleted"' $false$ ist, wird der Datensatz als neue Zeile eingefügt.

Für Parquet-Dateien werden die Änderungsdaten über Spark mit den Bestandsdaten zusammengeführt.
Beide Datensätze werden über einen Full-Outer-Join verknüpft.
Danach werden alle Zeilen, in denen das Feld "`cd\_deleted"' den Wert $true$ enthält, gelöscht.
Ähnlich zu der Deltaberechnung werden die übrigen Daten wieder auf das original Schema gebracht.
Dabei werden für alle Zeilen, bei denen entweder kein Eintrag aus den Bestandsdaten existiert oder die Änderungsdaten sich von ihnen unterscheiden, die Werte der Änderungsdaten übernommen.
In den anderen Zeilen werden die Bestandsdaten beibehalten.

\subsection{Plugin-Management}

Da Python verwendet wird, kann nicht, wie zum Beispiel in Java, ein Interface definiert werden, dass ein Plugin implementieren muss.
Es ist aber möglich die Namen, Parameter und Rückgabetyp einer Methode zu überprüfen.
Mit diesem Ansatz können Methoden aus den hochgeladen Plugin-Dateien validiert werden.
Für die Ingestion werden diese zwei Methoden definiert: \begin{enumerate}
    \item Load-Methode: \begin{verbatim}
        Name: "load"
        Rückgabetyp: DataFrame
        Parameter:
            - Name: "spark"
              Typ: SparkSession
    \end{verbatim}
    \item After-Load-Methode: \begin{verbatim}
        Name: "after_load"
        Rückgabetyp: DataFrame
        Parameter:
            - Name: "dataframe"
              Typ: DataFrame
    \end{verbatim}
\end{enumerate}

Für jede Ingestion wird auf dem Speichersystem des Mircoservices ein temporärer Ordner angelegt, in den die Plugin-Dateien abgelegt und deren Abhängigkeiten installiert werden.
Nach der Installation wird noch eine Datei angelegt, die für alle Pakete die Versionsnummern enthält.
Das dient dazu, bei einer erneuten Ausführung auf dem Service nicht alle Pakete neu installieren zu müssen, sondern nur die mit einer geänderten Versionsnummer.
Das beschleunigt die Ausführung der Ingestion.
Zum Schluss wird der Ordner, mit den installierten Paketen, zum Python-Pfad hinzugefügt.
Damit wird dieser während der Ausführung manipuliert und die Pakete sind verfügbar.
Da jede Ingestion in einem eigenen Prozess ausgeführt wird, beeinflussen die Änderungen an dem Python-Pfad den Ingestion-Service oder andere Prozesse nicht.

Im zweiten Schritt wird jede Plugin-Datei als Python-Modul geladen.
Jedes Modul enthält dann die in der Plugin-Datei definierten Methoden.
Diese Methoden können mit den Definitionen verglichen werden.
Dazu wird der \cref{algo:check-method} verwendet.
Diesem werden das geladene Modul, ein Name der Methode, ein optionaler Rückgabetyp der Methode und eine Liste von Parametern, bei denen Name und Typ definiert sind.
Zur Überprüfung können alle Methoden in dem Modul auf ihren Namen geprüft werden.
Wenn eine Methode gefunden wurde, wird eine Signatur erzeugt und mit der übergebenen Definition verglichen.
Das Ergebnis sagt dann, ob diese Methode ein Plugin ist oder nicht.

Jedes geladene Modul wird auf die Load- oder AfterLoad-Methoden geprüft.
Eine gefundene Load-Methode überschreibt immer die vorher gefundene, da jede Ingestion nur einen Weg zum Laden der Daten haben darf.
Die After-Load-Methoden dagegen werden in einer Liste gespeichert.
Es kann jedoch immer nur eine After-Load-Methode pro Plugin-Datei geben.
Diese können dann auch hintereinander ausgeführt werden, um auf einem DataFrame mehrere Modifikationen auszuführen.
Alle gefunden Methoden werden in einem Plugin-Manager gespeichert, um während der Ingestion aufgerufen werden zu können.
Die Reihenfolge der Ausführung entspricht der Reihenfolge, wie die Plugin-Dateien bei der Erstellung in der Liste angegeben wurden.

\begin{algorithm}
    \caption{Pluginmethode überprüfen}
    \label{algo:check-method}

    \textbf{Input:} $plugin$: Python-Modul, $name$: String, $return\_type$: Typ, $parameters$: Liste \\
    \textbf{Output:} $matches$: Boolean \\

    \If{$plugin$ \textbf{has no} method $name$}{
        \Return{false}
    }

    $signature \gets$ signature of mehtod $name$

    \If{$signature.return\_type$ \textbf{is not} $return\_type$}{
        \Return{false}
    }

    \ForAll{$param$ \textbf{in} $parameters$}{
        \If{$signature$ \textbf{has no} parameter $param.name$}{
            \Return{false}
        }
        \If{$signature.parameter.type$ \textbf{is not} $parameter.type$}{
            \Return{false}
        }
    }

    \Return{true}

\end{algorithm}

\subsection{Ausführung der Ingestion}
\label{sec:ingestion-run}
Die Ausführung startet mit der Initialisierung, welche die für die Ingestion benötigte Daten lädt.
Das ist zum Beispiel die DatasourceDefinition zu der Id aus den Events.
Danach wird entschieden, ob eine Ingestion über Spark notwendig ist.
Hier spielt der Lesetyp eine große Rolle.
Bei der Ingestion von unstrukturierten Daten wird Spark nicht benötigt.
Hier können die Dateien, über die WebHDFS-Schnittstelle direkt im HDFS verschoben werden.
Für alle anderen Typen wird im nächsten Schritt die Ingestion vorbereitet.
Es werden die Plugins aus dem HDFS geladen und eine SparkSession erstellt.
Das Laden der Daten in ein DataFrame geschieht anschließend entweder über ein Plugin in oder das Standardvorgehen.
Falls auch After-Load-Plugins vorhanden sind, werden diese ausgeführt.
Für die geladenen Daten wird dann entschieden, ob es sich um Änderungsdaten handelt oder welche berechnet werden müssen.
Je nach Schreib-Typ werden dann die Daten beziehungsweise Änderungsdaten gespeichert.
Für Datenströme wird am Ende noch ein Hintergrund Task gestartet, der auf Kafka Events zum Stoppen dieser Ingestion wartet.
Hierfür wird die Topic "`dls\_\_ingestion\_\_stop\_ingestion"' mit der Id als Wert verwendet.
Wenn die Ingestion beendet wurde, wird zum Schluss die SparkSession gestoppt.

\begin{figure}
    \centering
    \includegraphics[width=\textwidth]{Grafiken/Umsetzung-Ingestion-Ablauf.pdf}
    \caption{Ablauf einer Ingestion}
    \label{fig:umsetz-ingestion-ablauf}
\end{figure}

\chapter{Evaluierung}
\chapter{Ausblick}

\newpage

\printbibliography

\end{document}
