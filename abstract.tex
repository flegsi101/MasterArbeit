\section*{Zusammenfassung}

Heutzutage spielen Daten eine immer wichtigere Rolle.
Durch den vermehrten Einsatz von IoT-Geräten und moderner Cloud-Speicher-Lösungen, wächst die Zahl an anfallenden Daten in vielen Firmen und Forschungseinrichtungen stetig.
Durch steigenden Datenmengen und der diversen Strukturen der Daten ist deren Verwaltung ein komplexes Thema geworden.
Diese Arbeit befasst sich mit der technischen Herausforderung Daten aus verschiedensten Quellen zu Verwalten.
Als Lösung hierfür wurden Data-Lake-Systeme vorgeschlagen.
Im Kontext des HIT-Institut der Hochschule Niederrhein wurde ein Prototyp für einen Data-Lake entwickelt.
In dieser Arbeit wird eine Schnittstelle entwickelt, über die Benutzer Daten aus unterschiedlichen Quellen in dieses System laden können.
Dabei werden Metadaten über die Datenquellen gesammelt.
Mit der Schnittstelle ist es auch möglich, die geladenen Daten zu versionieren, um weiter Verarbeitungen effizienter zu machen.

\section*{Abstract}

In todays world data play an important role.
The amount of data in comapnies and research facilities is growing due to the increasing use of IoT-devices and modern Cloud-Storage-Solutions.
With the bigger amount of data and their varying structures the data management got more complex.
This thesis takes on the technical challenge of managing data from different sources.
Data lakes are a proposed solution for this problem.
A prototype for a data lake was developed at the HIT-institute at the Hochschule Niederrhein.
This thesis devolopes an interface for users to ingest data from diffenrent sources to the system.
The interface collects metadata about the data sources.
It is also capable of saving data with versioning to make further processing more efficient.