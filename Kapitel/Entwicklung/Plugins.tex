\section{Plugins}

In \ref{ANF_04} wird durch ANF\_04 gefordert, dass zusätzlicher Code bei der Ingestion ausgeführt werden können soll.
Das soll durch Plugins umgesetzt werden.
Die Plugins werden jeder Datenquelle einzeln hinzugefügt und werden an verschiedenen, fest definierten Punkten der Ingestion ausgeführt. 
Da die Plugins eventuell auf Software-Bibliotheken zurückgreifen müssen, die nicht auf dem Data-Lake-System vorhanden sind, kann zusätzlich eine Liste von Abhängigkeiten der Plugins angegeben werden.

Bei der Ingestion gibt es zwei Stellen, an denen das Einbringen eines Plugins sinnvoll sein kann.
Die erste ist zum Laden der Daten als Load-Plugin.
Hier wird das standardmäßige Vorgehen der Ingestion mit dem des Plugins ersetzt.
Dadurch wird es möglich anders als nur über \textit{Apache Spark} Daten zu laden.
Ein Beispiel dafür ist die Verwendung einer REST-API als Datenquelle.
Das Plugin kann erst über mehrere Abfragen der REST-API den Datensatz abholen und diesen dann erst in eine DataFrame umwandeln.
Ein Load-Plugin muss immer eine DataFrame zurück geben, mit dem danach in der Ingestion weiter verfahren werden kann.
Damit ein DataFrame erstellt werden kann,  muss dem Plugin außerdem die entsprechende SparkSession mitgegeben werden.

Das After-Load-Plugin setzt im Gegensatz direkt nach dem Laden der Daten an.
Diesem Plugin wird das geladene DataFrame übergeben und es muss auch wieder ein DataFrame zurück geben.
Es kann genutzt werden um vor dem Speichern der Daten kleinere Anpassungen am Datensatz zu machen.