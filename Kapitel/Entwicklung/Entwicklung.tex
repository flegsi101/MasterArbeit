\chapter{Entwicklung}
In diesem Kapitel werden die einzelnen Komponenten, die für eine komplette Ingestion notwendig sind entwickelt.
Nach \citeauthor{DL-Ing-Mgmt} ist der Ingestion-Prozess eines Data-Lake-Systems, als erster Schritt im Lebenszyklus der Daten, maßgebend dafür, wie gut die Daten später verwendet und verarbeitet werden können.
Um diese Aufgabe so gut wie möglich zu erfüllen, wurden im vorherigen Kapitel bereits die Anforderung festgelegt.
Daraus ergeben sich die drei Inhaltlichen Bereiche, nach denen die weiter Entwicklung aufgeteilt ist.

Der erste ist die Ingestion.
Diese beschreibt das erfassen von Informationen über eine Datenquelle und das Laden der Daten aus der Datenquelle.
Dabei werden die Daten noch nicht intern im Data Lake wieder abgelegt sondern existieren nur flüchtig im Arbeitsspeicher.

Die Deltaerkennung ist dafür zuständig die Unterschiede zwischen den geladenen Daten im Arbeitsspeicher und den Bestandsdaten aus dem internen Data Lake Speicher zu finden.
Aus diesen Unterschieden sollen Änderungsdaten erstellt werden, die dann bei der weiteren Verarbeitung verwendet werden können.

Der dritte Bereich ist das Speichern von Daten.
Als Zielspeicher gibt es hier einmal ein vordefiniertes internes Speichersystem, es ist aber auch auch möglich ein benutzerdefiniertes zu verwenden.
Der Unterschied dabei ist, dass im internen Speicher die Datenversionierung durch das Data-Lake-System unterstützt wird, während dies bei benutzerdefinierten nicht der Fall ist.

\section{Architektur}
\label{sec:arch}

Zu Anfang ist es sinnvoll, die Architektur des Systems festzulegen.
Diese bestimmt, welche Komponenten und Microservices entwickelt und verbunden werden müssen.
Der erste Schritt dabei ist es, die Aufgaben aufzuteilen, die das System erfüllen soll.
Diese Aufgaben werden dann auf Microservices verteilt.
Die weitere Entwicklung des Systems stützt sich dann auf die fertige Architektur.

\subsection{Aufgabenverteilung}
Durch die Verwendung von \textit{Apache Spark} ist es nicht sinnvoll, dass Laden der Daten, die Deltaerkennung und das Speichern zu trennen.
In \textit{Apache Spark} werden alle drei Arbeitsschritte auf einem DataFrame ausgeführt.
Das heißt, dass dieses zwischen den Microservices ausgetauscht werden müsste, was zu einem erheblichen Aufwand führen würde.
Aus diesem Grund kann die logische Aufteilung in Ingestion, Deltaerkennung und Speicherung nicht auch als Aufgabenverteilung verwendet werden.

Besser trennbare Aufgaben findet man bei der Betrachtung der technischen Seite.
Hierbei gibt es die API, die Ingestion in \textit{Apache Spark} und das kontinuierliche Ausführen.

Bei der \textbf{API} handelt es sich um den Service für die Interaktion mit dem Data-Lake-System.
Durch die Anforderungen ist bereits festgelegt, dass dieser ein Web-Server mit einer REST-Schnittstelle ist.
Es geht zwar in dieser Arbeit nur um die Ingestion, aber der API-Service sollte Schnittstellen zu allen Funktionen des Data-Lake-Systems enthalten.

Die \textbf{Ingestion} ist dafür zuständig, die Datenquellen zu verarbeiten und den kompletten Prozess vom Laden bis zum Speichern der Daten in \textit{Apache Spark} auszuführen.
Die Ausführung soll für eine Datenquelle nur einmal gleichzeitig aber parallel für unterschiedliche laufen.

Bei einer zeitgesteuerten oder Datenstrom-Ingestion muss die \textbf{kontinuierliche Ausführung} sichergestellt werden.
Für alle Datenquellen muss regelmäßig geprüft werden, ob für diese gerade eine Ingestion ausgeführt wird und ausgeführt werden sollten.
Falls keine ausgeführt wird aber sollte, wird die Ingestion für diese Datenquelle gestartet.

\subsection{Komponenten und Microservices}
Die drei oben genannten Aufgaben können jeweils einem Microservice zugeordnet werden.
In \fref{fig:ingestion_arch} ist eine dazu passende Architektur zu sehen.
Der API-Service übernimmt die REST-Schnittstelle, der Ingestion-Service kümmert sich um die Ausführung der Ingestion und der Continuation-Service stellt die kontinuierliche Ausführung sicher.

Neben diesen Mircoservices wird noch ein Nachrichtensystem benötigt.
Das Nachrichtensystem stellt die Kommunikation zwischen den Microservices dar.
Hier ist es wichtig, dass es einem Sender möglich ist, Nachrichten an einen oder auch an mehrere Empfänger zu senden.
So soll sichergestellt werden, dass bestehende Microservices einfach repliziert und neue eingefügt werden können.

Zum Ablegen von internen Informationen wird eine Datenbank benötigt.
Alle Microservices haben zugriff auf diese Datenbank und können Daten in ihr bearbeiten.
Es handelt sich dabei aber nicht um den internen Speicher für geladene Daten des Data-Lake-Systems sondern nur um Daten, die für den Betrieb des Systems benötigt werden.
Darunter fallen zum Beispiel Authentifizierungsdaten oder Verbindungsinformationen von Datenquellen.

\begin{figure}
  \centering
  \includegraphics{Grafiken/ingestion-arch.pdf}
  \caption{Architektur der Ingestion Komponenten}
  \label{fig:ingestion_arch}
\end{figure}

\section{Plugins}

In \nameref{ANF_04} wird gefordert, dass zusätzlicher Code bei der Ingestion ausgeführt werden können soll.
Das kann durch Plugins umgesetzt werden.
Plugins können einer Datenquelle hinzugefügt und an verschiedenen, fest definierten Punkten ausgeführt werden.
Da die Plugins eventuell auf Software-Bibliotheken zurückgreifen müssen, die nicht auf dem Data-Lake-System vorhanden sind, kann zusätzlich eine Liste von Abhängigkeiten der Plugins angegeben werden.
Das Prinzip der Plugins kann beim Ausbau auch über die Ingestion hinaus im System angewendet werden.

Für die Ingestion gibt es zwei Stellen, an denen die Möglichkeit für Plugins gegeben sein sollte.
Die erste Möglichkeit ist, wie durch die Anforderung gefordert, das Laden von Daten.
Genauer bedeutet das, dass das Plugin die Aufgabe übernimmt das DataFrame zu erstellen, welches später wieder gespeichert wird.
Das Plugin ersetzt die normale Funktion zum Laden in ein DataFrame.
Dies wird zum Beispiel bei der Ingestion von Daten aus einer REST-API benötigt.
Hier muss ein DataFrame manuell aus Daten erzeugt werden, die erst über einen REST-Client abgefragt werden.

Als zweite Möglichkeit sollte es Plugins geben, mit denen man nach dem Laden den DataFrame manipulieren kann.
Streng genommen widerspricht diese Möglichkeit dem Data-Lake-Prinzip, Daten unverändert in ihrem Rohzustand zu speichern.
Allerdings ist zum Beispiel bei der Anbindung von Kafka-Datenströmen eine solche Nachbearbeitung sinnvoll, da die Daten nur als Byte-Block übertragen werden.
Für solche Fälle sollte es möglich sein, die unstrukturierten Byte-Daten in ein strukturiertes Format zu überführen.
\section{Datenquellen}
Ein weiterer Kernpunkt der Ingestion-Schnittstelle ist die Modellierung der Datenquellen.
Zur Entwicklung eines Datenmodells für die Datenquellen, wird als erstes betrachtet, welche verschiedenen Typen von Datenquellen möglich sind.
Danach wird unter Zuhilfenahme des Ergebnisses ein Modell entwickelt.
Das fertige Datenmodell enthält dann neben den für den Betrieb benötigten Informationen auch Daten über den Verlauf der Ingestion einer Datenquelle.
Um den Verlauf von Änderungen an einer Datenquelle möglich zu machen werden alle Überarbeitungen in Revisionen festgehalten.

\subsection{Datenquellen-Typen}
\label{sec:datasource-types}

Da das System alle möglichen Datenquellen unterstützen soll, werden hier mögliche Typen dargestellt, mit denen sich alle Datenquellen abdecken lassen.
Durch die Verwendung von \textit{Apache Spark} spielt bei der Unterscheidung der Daten quellen die Struktur keine Rolle.
Es können sowohl strukturierte als auch semi- oder unstrukturierte Daten verarbeitet werden.

Die Typen der Datenquellen ergeben sich aus der Betrachtung, wie die Daten in das Data-Lake-System gelangen.
Bei dem Pull-Prinzip ist das System dafür verantwortlich Daten aus einer Quelle zu laden.
Das Gegenteil dazu ist dass Pull-Prinzip.
Bei diesem werden Daten direkt an das System gesendet.
Das Push-Prinzip lässt sich dabei noch weiter aufteilen.
Es gibt als erstes Datenströme, bei denen kontinuierlich laufend neue Daten an das System gesendet werden.
Als zweites gibt es Dateien, die als Datenquelle dienen.
Es wird bewusst auf die Möglichkeit verzichtet ganze Datensätze direkt an das Data-Lake-System zu senden, da es einfacher ist, eine große Menge von Daten auf ein oder mehrere Dateien zu verteilen und diese an System zu übergeben.

Daraus ergeben sich verschiedene Typen nach denen die Datenquellen aufgeteilt werden können.
Diese sind Pull, Datei und Datenstrom.
Bei der Verarbeitung können eine Pull- und eine Datei-Ingestion gleich behandelt werden.
Für beide besteht die Möglichkeit die Ingestion nur einmalig auszuführen oder als kontinuierliche Ingestion über manuelle oder zeitgesteuerte Auslösung.
Datenströme hingegen sind immer kontinuierlich.

\subsection{Datenquellen-Modell}
\label{sec:datasourcemodel}

Das Modell einer Datenquelle besteht aus drei Teilen.
Eine Liste von Ingestion-Events, die Informationen über jede ausgeführte Ingestion enthalten.
Die Revisionen, die alle Informationen zur Datenquelle beinhalten, die durch den Benutzer geändert werden können.
In einer Datenquelle werden diese beiden Listen zusammen mit unveränderlichen Informationen wie einer Id aggregiert.

Im folgenden werden die Felder der Modelle beschrieben.
Falls eine Information direkt aus \textit{Apache Spark} abgeleitet ist, wird auch Angegeben an welcher Stelle diese angewendet wird.

\subsubsection*{Revision} 
    
\paragraph{Nummer} 
Eine aufsteigende Nummer, die die Revision identifiziert.
Sie startet bei 0 und jede Nummer ist für eine Datenquelle einzigartig.
Gleichzeitig spiegelt die Nummer auch den zeitlichen Verlauf wieder. 

\paragraph{Erstellungsdatum} 
Das Datum an dem die Revision erstellt wurde.
Da eine Revision nicht geändert werden kann, ist das Erstellungsdatum einer Revision auch das Datum der Änderung der dazugehörigen Datenquelle.

\paragraph{Name} 
Ein Name, der der Datenquelle im Datalake gegeben wird.
Dieser dient nur dazu die Datenquelle als Benutzer besser identifizieren zu können.

\paragraph{Id Spalte} 
Der Name der Spalte oder des Feldes, das einen Datensatz eindeutig identifiziert.
Die Id Spalte wird für die Zuordnung der Datensätze bei der Deltaerkennung benötigt.

\paragraph{Spark Abhängigkeiten} 
Eine Liste von Abhängigkeiten, die der SparkSession bei der Ingestion mitgegeben wird.
Diese Liste entspricht der Option "`spark.jars.packages"' die bei der Erstellung einer Spark Session gesetzt wird.

\paragraph{Quelldateien} 
Bei der Ingestion können die Daten über das Push-Prinzip in Form von Dateien an den Server gesendet werden.
Die Liste der Quelldateien enthält die Pfade zu diesen Dateien.

\paragraph{Typ beim Lesen} 
Der Typ der zu lesenden Daten, wie in \ref{sec:datasource-types} beschrieben.

\paragraph{Format beim Lesen} 
Das Format in dem die Daten gelesen werden sollen.
Der Wert aus diesem Feld wird bei der Ingestion über die \verb|format| Methode eines 
Readers gesetzt.

\paragraph{Optionen beim Lesen} 
Eine Liste von Schlüssel-Wert-Paaren, die als Optionen des Readers in \textit{Apache Spark} gesetzt werden.
Über diese wird die Verbindung zur Datenquelle definiert.

\paragraph{Aktualisierungsquelle} 
Es kann für einen Datenquelle festgelegt werden, dass diese Aktualisierungen für eine andere enthält.
Dann wird sie Aktualisierungsquelle genannt.
Das ermöglicht die Nutzung von eigenen Change Data Capture Lösungen.
Dabei muss die Datenquelle aber dem einem festgelegten Anderäungsdatenformat entsprechen, das später genauer erläutert wird.
Das Feld selbst enthält die Id der Ziel-Datenquelle für die Aktualisierung.

\paragraph{Typ beim Schreiben} 
Der Typ für das Schreiben der Daten legt fest, ob intern mit Versionierung oder in einen freien Speicher geschrieben werden soll.

\paragraph{Format und Optionen beim Schreiben}
Das Format und die Optionen beim Schreiben funktionieren analog zu denen beim Lesen.
Sie werden jedoch nicht beim Reader gesetzt sondern beim Writer.
Gerade bei Datenströmen muss darauf geachtet werden, dass nicht in jedem Format ein Datenstrom geschrieben werden kann.

\paragraph{Schreibmodus} 
Der Modus in dem die Daten geschrieben werden.
Die Auswahlmöglichkeiten werden auch durch \textit{Apache Spark} festgelegt.

\paragraph{Zeitsteuerung} 
Eine Liste die festlegt, zu welchen Zeitpunkte eine Ingestion der Datenquelle ausgeführt werden soll.

\paragraph{Plugin Abhängigkeiten} 
Eine Liste von Abhängigkeiten, die von den Plugins der Datenquelle benötigt werden.

\paragraph{Plugins} 
Die Speicherorte der Plugins. 
    
\subsubsection*{Ingestion-Event}

\paragraph{Nummer}
Die Nummer um ein Ingestion-Event zu identifizieren.
Sie funktioniert genau wie die Nummer der Revision. 

\paragraph{Status}
Der aktuelle Status in dem sich das Event befindet.
Er gibt Auskunft, ob die Ingestion gestartet wurde, gerade läuft oder beendet wurde. 

\paragraph{Revisionsnummer}
Die Nummer der Revision, mit der die Ingestion gestartet wurde.
Mit Hilfe der Revisionsnummer ist es leichter Fehler in der Ingestion zu finden und zu beheben. 

\paragraph{Start- und Endzeit}
Die Zeitpunkte des Starts und Endes eines Ingestion-Durchlaufs.
Wenn die Ingestion noch in der Ausführung ist, ist keine Endzeit gesetzt. 

\paragraph{Fehler}
Die Fehlerausgabe, wenn die Ingestion nicht erfolgreich ausgeführt werden konnte. 
    
\subsubsection*{Datenquelle} 
    
\paragraph{Id} 
Eine Identifikationsnummer, die für jede Datenquelle eindeutig ist. 

\paragraph{Aktuelle Revision}
Die Nummer der aktuellen Revision.
Neue Ingestions werden mit den Informationen aus dieser Revision ausgeführt. 

\paragraph{Alle Revisionen}
Eine Liste aller Revisionen. 

\paragraph{Letzte Ingestion}
Die Nummer des zuletzt gestarteten Ingestion-Durchlaufs. 

\paragraph{Letzte erfolgreiche Ingestion}
Die Nummer des letzten erfolgreich abgeschlossenen Ingestion-Durchlauf. 

\paragraph{Alle Ingestion-Events}
Eine Liste aller Ingestion Events.


\section{API-Service}

Der API-Service benötigt nicht viel Entwicklung.
Es müssen lediglich die Enspunkte definiert werden, die Zugriff auf die verschiedenen Funktionen der Ingestion-Schnittstelle bereitstellen.
Dazu gehören die Verwaltung von Datenquellen und das Starten einer Ingestion.
Da es sich um eine REST-Schnittstelle handelt, werden die Endpunkte hier durch einen Pfad und eine HTTP-Methode definiert (\fref{tab:endpunkte}).
Außerdem kümmert sich der API-Service um die Erstellung von DatasourceDefinitions und deren Revisionen und IngestionEvents.
Bei Anfragen zum Starten einer Ingestion versendet der API-Server eine Nachricht mit der Id einer Definition der zu ladenen Datenquelle.

    {\renewcommand{\arraystretch}{1.8}
        \begin{table}[ht]
            \centering
            \begin{tabularx}{\linewidth}{|lX|}
                \hline
                GET  & /datasources                                                                         \\
                \multicolumn{2}{|l|}{Liefert alle im System gespeicherten Datenquellen}                     \\
                \hline
                GET  & /datasources/\textless id\textgreater                                                \\
                \multicolumn{2}{|l|}{Liefert die Datenquelle mit der im Pfad übergebenen Id}                \\
                \hline
                POST & /datasources                                                                         \\
                \multicolumn{2}{|l|}{Bearbeitet die Daten Datenquelle mit der im Pfad übergebenen Id}       \\
                \hline
                PUT  & /datasources/\textless id\textgreater                                                \\
                \multicolumn{2}{|l|}{Erstellt eine neue Datenquelle}                                        \\
                \hline
                GET  & /datasources/\textless id\textgreater/run                                            \\
                \multicolumn{2}{|l|}{Startet eine Ingestion der Datenquelle mit der im Pfad übergebenen Id} \\
                \hline
            \end{tabularx}
            \caption{Endpunkte des API-Servers}
            \label{tab:endpunkte}
        \end{table}
    }

\section{Continuation-Service}

Für die Sicherstellung der korrekten Ausführung kontinuierlicher Ingestions, müssen regelmäßig alle Datenquellen überprüft werden.
Dabei gibt es zwei Bedingungen nach denen entscheiden wird, ob eine Ingestion ausgeführt werden muss.
Bei Datenströmen gilt allgemein, wenn dieser nicht läuft, dann muss die Ingestion automatisch neu gestartet werden.
Die einzige Ausnahme ist, wenn die Ingestion durch den Benutzer explizit beendet wurde.

Der zweite Fall ist die Zeitsteuerung.
Für eine zeitgesteuerte kontinuierliche Ingestion werden einer Datenquelle ein oder mehrere Timer hinzugefügt.
Der Continuation-Service kontrolliert, ob der Timer zum Zeitpunkt, an dem die Datenquelle überprüft wird zutrifft oder nicht.
Wenn das der Fall ist und bisher keine Ingestion auf der Datenquelle läuft, dann wird eine neue Ingestion gestartet.

In Unix-Systemen gibt es bereits eine Lösung für die Notation solcher Timer.
Dort gibt es die sogenannten Cron-Jobs, mit deren Hilfe Aufgaben automatisch und regelmäßig ausgeführt werden können.
Dabei wird der Zeitpunkt der Ausführung über fünf Felder festgelegt.
Diese geben die Minute, die Stunde, den Tag des Monats, den Monat und den Tag der Woche als Zahlen an.
Als Erweiterung kann man "`*"' als Platzhalter für alle möglichen Werte verwenden, man kann mehrere Werte als mit Kommata getrennt angeben oder mit "`/x"' eine Liste in Schritten der Größe $x$ erzeugen.

Diese Notation soll auch für die Zeitsteuerung der Ingestions genutzt werden.
Als Referenz wird dabei die koordinierte Weltzeit (UTC) genommen, damit die Ausführung unabhängig vom Standort einheitlich bleibt.
Wenn eine Datenquelle mehrere Timer hat, reicht es aus, dass einer von diesen zutrifft.
\section{Ingestion-Service}
\label{sec:entw-ingestion}

Der Ingestion-Service hat die Aufgabe eine Ingestion für eine Datenquelle durchzuführen.
Dazu gehören das Laden der Daten in ein DataFrame, die Deltaberechnung und das Speichern.
Das meiste davon wird jedoch nicht von dem Service selbst, sondern auf dem Spark Cluster gemacht.
Der Service führt die Vorbereitung und das Deplyoment des Jobs auf dem Cluster aus.

Der Ingestion-Service wartet auf die Nachricht zur Ausführung einer Ingestion, mit der Id der DatasourceDefinition.
Als erstes wird dann geprüft, ob bereits eine Ingestion der Datenquelle aktiv ist.
Falls das nicht der Fall ist, wird ein neuer Prozess gestartet, indem die Ingestion ausgeführt wird.
Auf diese Art können mehrer Ingestion von verschiedenen Quellen parallel bearbeitet werden.

Der Ablauf einer einzelnen Ingestion kann unabhängig vom Lese- und Schreib-Typ in einem allgemeinen Ablauf abgebildet werden.
Als erstes wird die Ingestion vorbereitet.
Hier werden die Plugins installiert und eine SparkSession erstellt.
Im nächsten Schritt werden die Daten aus der Quelle geladen.
Wenn es sich dabei um Ändeurngsdaten aus einer Updatequelle handelt, können diese direkt in den entsprechenden Zieldatensatz eingepflegt werden.
Ist das nicht der Fall, foglt eine Entscheidung, ob Änderungsdaten berechnet werden müssen.
Es gelten die folgenden zwei Regeln: \begin{itemize}
    \item der Speicher-Typ ist Delta
    \item es ist nicht die erste Ingestion dieser Datenquelle
\end{itemize}
Wurden Änderungsdaten berechnet, werden diese eingepflegt und ansonsten einfach gespeichert.
Handelt es sich nicht um einen benutzerdefinierten Speicher, werden die alten Daten mit den neuen überschrieben.

\begin{figure}
    \centering
    \includegraphics[width=\textwidth]{Grafiken/Entwicklung-Ingestion-Ablauf.pdf}
    \caption{Ablauf einer Ingestion}
    \label{fig:ingestion-ablauf}
\end{figure}