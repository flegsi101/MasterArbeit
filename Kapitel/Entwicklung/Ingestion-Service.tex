\section{Ingestion-Service}

Der Ingestion-Service hat die Aufgabe eine Ingestion für eine Datenquelle durchzuführen.
Dazu gehören wie in der Einleitung des Kapitels bereits genannt, das Laden der Daten, die Deltaerkennung und das Speichern.
Damit ein Ingestion-Service mehrere Ingestions zu verschiedenen Datenquellen gleichzeitig durchführen kann, wartet dieser auf Benachrichtigungen der anderen Services, dass eine Ingestion gestartet werden soll.
In dieser Nachricht wird auch die Id der Datenquelle übertragen.
Dann lädt der Ingestion-Service die Datenquelle aus der internen Datenbank und prüft ob eine Ingestion für diese bereits läuft.
Wenn das nicht der Fall ist, wird die Ingestion in einem neuen Prozess gestartet.

\subsection{Ablauf}
Der Ablauf einer einzelnen Ingestion kann über alle Typen gleich etwas abstrahiert gezeigt werden.
Im ersten Schritt müssen die Plugins für die Ingestion geladen werden.
Dabei ist es wichtig auch eventuelle Abhängigkeiten von Bibliotheken aufzulösen.

Danach können die Daten entweder über das Plugin oder normal in ein Dataframe geladen werden.
Ingestions von Dateien oder nach dem Pull-Prinzip können gleich abgehandelt werden.
Nur die Pfade zu den Dateien müssen durch den Service ergänzt werden.
Eine Ingestion eines Datenstroms hat eine eigene Behandlung.

Im Anschluss wir geprüft, ob eine Deltaerkennung für die Daten durchgeführt wird.
Dazu müssen drei Bedienungen erfüllt sein.
Es darf sich nicht um eine Datenquelle handeln, für die ein benutzerdefinierter Speicher festgelegt wurde.
Für diese kann keine Versionierung unterstütze werden.
Außerdem ist es für Aktualisierungsquellen nicht nötig, eine Deltaerkennung  durchzuführen, da diese schon im Änderungsdatenforamt sein sollen.
Für den letzten Punkt muss es sich um die wiederholte Ingestion einer Datenquelle handeln, da bei der ersten keine Bestandsdaten für eine Deltaerkennung existieren.

Zum Schluss wird auch beim Speichern der Daten unterschieden.
Bei benutzerdefinierten Speichern können die Daten wie konfiguriert ohne weiteren Aufwand gespeichert werden.
Das gilt für sowohl Datenströme als auch andere Ingestions.
Änderungsdaten müssen in eine existierenden Speicher integriert werden.
Der letzte Fall bedeutet, dass Daten das erste mal im internen Speichersystem gespeichert werden und auch hier in einer neuen Datensatz abgelegt werden können.