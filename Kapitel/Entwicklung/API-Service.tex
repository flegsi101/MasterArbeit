\section{API-Service}

Der API-Service benötigt nicht viel Entwicklung.
Es müssen lediglich die Enspunkte definiert werden, die Zugriff auf die verschiedenen Funktionen der Ingestion-Schnittstelle bereitstellen.
Dazu gehören die Verwaltung von Datenquellen und das Starten einer Ingestion.
Da es sich um eine REST-Schnittstelle handelt, werden die Endpunkte hier durch einen Pfad und eine HTTP-Methode definiert (\fref{tab:endpunkte}).
Außerdem kümmert sich der API-Service um die Erstellung von DatasourceDefinitions und deren Revisionen und IngestionEvents.
Bei Anfragen zum Starten einer Ingestion versendet der API-Server eine Nachricht mit der Id einer Definition der zu ladenen Datenquelle.

    {\renewcommand{\arraystretch}{1.8}
        \begin{table}[ht]
            \centering
            \begin{tabularx}{\linewidth}{|lX|}
                \hline
                GET  & /datasources                                                                         \\
                \multicolumn{2}{|l|}{Liefert alle im System gespeicherten Datenquellen}                     \\
                \hline
                GET  & /datasources/\textless id\textgreater                                                \\
                \multicolumn{2}{|l|}{Liefert die Datenquelle mit der im Pfad übergebenen Id}                \\
                \hline
                POST & /datasources                                                                         \\
                \multicolumn{2}{|l|}{Bearbeitet die Daten Datenquelle mit der im Pfad übergebenen Id}       \\
                \hline
                PUT  & /datasources/\textless id\textgreater                                                \\
                \multicolumn{2}{|l|}{Erstellt eine neue Datenquelle}                                        \\
                \hline
                GET  & /datasources/\textless id\textgreater/run                                            \\
                \multicolumn{2}{|l|}{Startet eine Ingestion der Datenquelle mit der im Pfad übergebenen Id} \\
                \hline
            \end{tabularx}
            \caption{Endpunkte des API-Servers}
            \label{tab:endpunkte}
        \end{table}
    }
