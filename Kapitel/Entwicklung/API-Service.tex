\section{API-Service}

Für den API-Service müssen die Endpunkte definiert werden, die den Zugriff auf die verschiedenen Funktionen der Ingestion-Schnittstelle geben.
Dazu gehören die Verwaltung von Datenquellen und das Starten einer Ingestion.
Da es sich um eine REST-Schnittstelle handelt, können die Endpunkte durch einen Pfad und eine HTTP-Methode definiert werden (\cref{tab:endpunkte}).
Außerdem kümmert sich der API-Service um die Erstellung von DatasourceDefinitions und deren Revisionen und IngestionEvents.
Hierbei ist nicht nur die Erstellung der Revisionen wichtig, sondern auch die Handhabung der Quelldateien.
Diese müssen so abgelegt und in der Revision verwaltet werden, dass Spark später bei der Ingestion darauf zugreifen kann, aber bereits hochgeladenen Dateien nicht gelöscht werden.
Bei Anfragen zum Starten einer Ingestion versendet der API-Server eine Nachricht mit der Id einer Definition der Datenquelle, die geladen werden soll.

    {\renewcommand{\arraystretch}{1.8}
        \begin{table}[ht]
            \centering
            \begin{tabularx}{\linewidth}{|lX|}
                \hline
                GET  & /datasources                                                                         \\
                \multicolumn{2}{|l|}{Liefert alle im System gespeicherten Datenquellen}                     \\
                \hline
                GET  & /datasources/\textless id\textgreater                                                \\
                \multicolumn{2}{|l|}{Liefert die Datenquelle mit der im Pfad übergebenen Id}                \\
                \hline
                POST & /datasources                                                                         \\
                \multicolumn{2}{|l|}{Bearbeitet die Daten Datenquelle mit der im Pfad übergebenen Id}       \\
                \hline
                PUT  & /datasources/\textless id\textgreater                                                \\
                \multicolumn{2}{|l|}{Erstellt eine neue Datenquelle}                                        \\
                \hline
                GET  & /datasources/\textless id\textgreater/run                                            \\
                \multicolumn{2}{|l|}{Startet eine Ingestion der Datenquelle mit der im Pfad übergebenen Id} \\
                \hline
            \end{tabularx}
            \caption{Endpunkte des API-Servers}
            \label{tab:endpunkte}
        \end{table}
    }
