\section{API-Service}

Für den API-Service müssen Endpunkte definiert werden.
Diese Endpunkte bilden die verschiedenen Funktionen ab, die auf der Ingestion-Schnittstelle ausgeführt werden können.
Dazu gehören die Verwaltung von Datenquellen und das Starten einer Ingestion.
Da es bereits festgelegt wurde, dass es sich um eine REST-Schnittstelle handelt, werden Endpunkte durch einen Pfad und eine HTTP-Methode definiert.
Nachfolgend werden alle Endpunkte aufgelistet.

\begin{table}[ht]
  \centering
  \begin{tabularx}{\linewidth}{lX}
    GET & /datasources \\
    \multicolumn{2}{l}{Liefert alle im System gespeicherten Datenquellen} \\
    \\
    GET & /datasources/\textless id\textgreater \\
    \multicolumn{2}{l}{Liefert die Datenquelle mit der im Pfad übergebenen Id} \\
    \\
    POST & /datasources \\
    \multicolumn{2}{l}{Bearbeitet die Daten Datenquelle mit der im Pfad übergebenen Id} \\
    \\
    PUT &  /datasources/\textless id\textgreater \\
    \multicolumn{2}{l}{Erstellt eine neue Datenquelle} \\
    \\
    GET &  /datasources/\textless id\textgreater/run \\
    \multicolumn{2}{l}{Startet eine Ingestion der Datenquelle mit der im Pfad übergebenen Id}
    
    
    
    %\hline
    %Pfad                                      & HTTP-Methode & Beschreibung                                                          \\
    %\hline \hline
    %/datasources                              & GET          & Liefert alle im System gespeicherten Datenquellen                     \\
    %\hline
    %/datasources/\textless id\textgreater     & GET          & Liefert die Datenquelle mit der im Pfad übergebenen Id                \\
    %\hline
    %/datasources                              & POST         & Erstellt eine neue Datenquelle                                        \\
    %\hline
    %/datasources/\textless id\textgreater     & PUT          & Bearbeitet die Daten Datenquelle mit der im Pfad übergebenen Id       \\
    %\hline
    %/datasources/\textless id\textgreater/run & GET          & Startet eine Ingestion der Datenquelle mit der im Pfad übergebenen Id \\
    %\hline
  \end{tabularx}
  %\caption{Endpunkte des API-Servers}
  \label{tab:enpoints}
\end{table}

Außerdem kümmert sich der API-Service um die Erstellung von Datenquellen, deren Revisionen und Ingestion-Events.
Bei Anfragen zum Starten einer Ingestion versendet der API-Server eine Nachricht, mit der Id der Datenquelle.