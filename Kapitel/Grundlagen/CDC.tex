\section{Change Data Capture}
\label{sec:cdc}

Um Änderungen an Daten in Datenquellen in das Data-Lake-System einpflegen zu können, müssen diese erst erfasst werden.
Diesen Prozess nennt man Change-Data-Capture (CDC).
Das Ziel beim CDC ist es, die Änderungen an den Daten nur an einer Stelle zu erfassen und dann an andere Systeme weiterzugeben, damit folgende Verarbeitungsschritte nur die Änderungen berücksichtigen und nicht auf den gesamten Datenbestand zurückgreifen müssen.
Dafür gibt es verschiedene Ansätze, die auf Datenbank-Triggern \parencite{boeing}, Log-Einträgen \parencite{delta-view_gen}, Zeitstempeln \parencite{delta-view_gen, boeing} oder Snapshots \parencite{cdc_in_nosql} basieren.

\subsection{Änderungserfassung Datenbank-Triggern}
Datenbank-Trigger sind Funktionen, die bei verschiedenen Aktionen auf den Daten in einer Datenbank ausgelöst werden.
Über diese Trigger lassen sich CDC-Programme realisieren, die Änderungen genau dann festhalten, wenn sie geschehen.
Ein Nachteil ist, dass die Methode nur in Systemen angewendet werden kann, die auch Trigger unterstützen.
Dafür ist es möglich alle Änderungen wie Einfügen, Aktualisieren oder Löschen von Daten zu erfassen \parencite{boeing}.

\subsection{Log-basierte Änderungserfassung}
Es gibt viele Datenspeicher-Systeme, die Logs über die Aktionen auf den Daten führen.
Diese werden zum Beispiel genutzt, um eine Wiederherstellung möglich zu machen.
Ein CDC-Programm kann diese Logs auslesen und daraus die Änderungsdaten erzeugen.
Hierdurch gibt es fast keinen zusätzlichen Aufwand für das eigentliche System.
Aber auch hier gilt, dass diese Methode davon abhängig ist, ob ein System Logs erstellt und diese durch externe Programme abgerufen werden können \parencite{delta-view_gen}.


\subsection{Zeitstempel-basierte Änderungserfassung}
Ein weiterer Ansatz ist die Verwendung von Zeitstempeln mit den Zeitpunkten der Erstellung und letzten Änderung.
Diese Zeitstempel müssen in jedem Datensatz vorhanden sein.
Die Verantwortung dafür kann entweder bei dem Ersteller der Daten liegen oder durch das Speichersystem automatisch hinzugefügt werden.
Das CDC-Programm überprüft regelmäßig alle Zeitstempel der Einträge in den Daten.
Wenn diese zwischen dem letzten und dem aktuellen Durchlauf liegen wird die Änderung erfasst.
Hierbei werden nur kumulierte Änderungen seit dem letzten Durchlauf erfasst.
Es ist nicht möglich nachzuvollziehen, welche und wie viele Änderungen in der Zeit gemacht wurden.
Außerdem lassen sich auch mit dieser Methode keine Löschungen erfassen \parencite{delta-view_gen}.
Der Aufwand für diese Methode kann relativ hoch werden, da ohne Indices auf den Zeitstempeln immer die gesamten Daten gelesen werden müssen \parencite{boeing}.

\subsection{Snapshot-basiert Änderungserfassung}
\label{sec:snaps}
Die letzte Methode ist das Vergleichen zweier Momentaufnahmen (Snapshots) eines Datensatzes.
Dabei wird bei jedem Durchlauf zuerst ein aktueller Snapshot generiert.
Dieser wird danach mit dem des vorherigen Durchlaufs verglichen, um alle Änderungen zu erhalten.
Hierfür muss ein separater Speicherort für die Snapshots festgelegt werden.
Wie bei den Zeitstempeln ist es nicht möglich den gesamten Änderungsverlauf zwischen zwei Snapshots nachzuvollziehen.
Außerdem müssen für den Vergleich immer alle Daten geladen werden, was zu einem hohen Rechen- und Speicheraufwand führen kann \parencite{cdc_in_nosql}.