\subsection{Delta Lake}

Als eine Lösung für die Versionierung von Daten gibt es den Delta Lake
Delta Lake ist eine extra Speicherebene, die auf dem HDFS oder eine Objektspeicher in der Cloud, wie Amazons S3, angewendet werden kann.
Das Ziel ist es, diesen Speichern ACID-Transaktionen, schnelles Arbeiten mit Metadaten der Tabelle und eine Versionierung der Daten hinzuzufügen.
Daten werden in sogenannten Delta-Tabellen mit Metadaten und Logs gespeichert.

Eine Delta-Tabelle wird zunächst durch ein Verzeichnis im Dateisystem dargestellt.
Die tatsächlichen Daten werden in diesem Verzeichnis als Parquet-Dateien abgelegt.
Dabei können die Daten auch noch in Unterverzeichnisse aufgeteilt werden, zum Beispiel für jedes Datum ein Verzeichnis.
Neben den Datenverzeichnissen gibt es in jeder Delta-Tabelle einen Ordner für die Logs in Form von JSON Dateien mit aufsteigender Nummerierung.
Metadaten werden sowohl innerhalb der Parquet- als auch in der Log-Dateien gespeichert.

Im Delta Lake wird ein Protokoll für den Zugriff verwendet, dass es mehreren Clients ermöglicht gleichzeitig Lesen zu können, aber immer nur einem das Schreiben erlaubt.
Dabei werden beim Schreiben immer erst neue Datensätze, die zur Tabelle hinzugefügt werden sollen, in das Verzeichnis der Delta-Tabelle geschrieben.
Danach wird eine neu Log-Datei erstellt.

Beim Lesen werden die Log-Dateien als Grundlage verwendet um daraus zusammen mit den gespeicherten Daten den Zustand der Tabelle an einem bestimmten Zeitpunkt zu erzeugen.
Standardmäßig wird beim Lesen immer die aktuellste Version verwendet, man kann aber auch eine bestimmte Version angeben.
Um den Aufwand bei der Verarbeitung der Logs zu verringern wird periodisch ein Kontrollpunkt erzeugt, bei dem alle vorherigen Logs zusammengefügt und komprimiert werden.
Das bedeutet, dass zum Beispiel das Operationen die sich gegenseitig aufheben nicht gespeichert werden.
Damit reicht es aus nur den letzten Checkpoint vor der zu lesenden Version und alle darauf folgende Logs zu lesen.

Durch das Design werden keine eigenen Server für die Pflege der Delta-Tabellen benötigt.
Diese Funktionen werden von den Clients übernommen.
Der Delta Lake unterstützt sowohl die Batch-Verarbeitung von Daten als auch Datenströme und bietet volle Integration in Spark \parencite{deltalake}.

