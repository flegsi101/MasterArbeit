\section{Ingestion}
Bei der Entwicklung der Ingestion wird das Laden der Daten und die Architektur der Anwendung betrachtet.
Es wird eine Datenmodell, in dem die Informationen zu den Datenquellen gespeichert werden, der Ablauf der verschiedenen Ingestion-Prozesse und die Kommunikation zwischen den einzelnen Mircoservices entwickelt.

\subsection{Server Architektur}

Für die Umsetzung der Mircoservice-Architektur wird die Ingestion in Komponenten aufgeteilt, die \fref{fig:ingestion_arch} zu sehen sind.
Es gibt drei Services, die für die Ingestion spezifischen Aufgaben zuständig sind, eine Datenbank, in der die Informationen über Datenquellen gespeichert werden und einen Service, der für die Kommunikation verantwortlich ist.
Der \textbf{Api-Server} bietet einen REST-Schnittstelle, über die man mit der Ingestion interagieren kann.
Hier werden die Endpunkte aus \fref{tab:enpoints} benötigt, die die Schnittstellen zur Verwaltung von Datenquellen und das Ausführen von Ingestions bereitstellen.
Außerdem ist er dafür zuständig, die empfangenen Informationen über Datenquellen in der Datenbank zu verwalten.
der \textbf{Continuation-Server} überprüft regelmäßig alle kontinuierlichen Datenquellen, ob diese eine Zeitsteuerung haben und aktuell ausgeführt werden sollten.
Der \textbf{Ingestion-Server} ist die Anwendung, die die eigentliche Ingestion ausführt.
Dafür wartet dieser auf eine Aufforderung durch entweder den Api- oder den Continuation-Server.

\begin{table}[!ht]
  \centering
  \begin{tabular}{| l | l | p{3in} |}
    \hline
    Pfad                                      & HTTP-Methode & Beschreibung                                                          \\
    \hline \hline
    /datasources                              & GET          & Liefert alle im System gespeicherten Datenquellen                     \\
    \hline
    /datasources/\textless id\textgreater     & GET          & Liefert die Datenquelle mit der im Pfad übergebenen Id                \\
    \hline
    /datasources                              & POST         & Erstellt eine neue Datenquelle                                        \\
    \hline
    /datasources/\textless id\textgreater     & PUT          & Bearbeitet die Daten Datenquelle mit der im Pfad übergebenen Id       \\
    \hline
    /datasources/\textless id\textgreater/run & GET          & Startet eine Ingestion der Datenquelle mit der im Pfad übergebenen Id \\
    \hline
  \end{tabular}
  \caption{Endpunkte des Api-Servers}
  \label{tab:enpoints}
\end{table}

\subsection{Plugins}
Da bei manchen Ingestions nicht immer ein festgelegtes Vorgehen ausreicht, um die Daten aus bestimmten Datenqellen zu laden, muss ein System entwickelt werden, wie möglichst ohne großen Aufwand die Inegstion erweitert werden kann.
Beispiele für solche Fälle sind die Verarbeitung von Datenströmen oder die Ingestion von Daten aus APIs, die nicht generallisiert werden können.
Als Lösung für das Problem, können Plugins der Datenquelle hinzugefügt werden.
Diese Plugins sollen Logik enthalten, die an verschiedenen Stellen der Ingestion ausgeführt werden sollen.
Aktuell sind diese Stellen das Laden der Daten, nach dem Laden der Daten und die Stream-Verarbeitung.
Dabei ist zu beachten, dass Plugins, die das Laden der Daten abhandeln, das Standardverhalten des Ingestion-Servers überschreiben und das Stream-Verarbeitung nur mit Plugins möglich ist.
Damit es nicht zu Konflikten bei Abhängigkeiten der Plugins gibt, muss der Ingestion-Server eine Mechanik implementieren, bei der die Abhängigkeiten der Plugins für jede Datenquelle dynamisch gealden werden.
