\section{Nachrichtensystem}

Für die Übermittlung von Nachrichten zwischen verschiedenen Anwendungen gibt es sogenannte Message-Broker.
Diese koordinieren als Mittelsmann die Verteilung der Nachrichten an verschiedene Empfänger.
Das hat den Vorteil, dass der Sender unabhängig von den Empfängern wird und die Kommunikation asynchron statt finden kann \parencite{message-broker}.

Es gibt mittlerweile einige Projekte, diese Aufgabe auf verschiedene Arten lösen.
Hier wird dafür Apache Kafka verwendet, welches im Big Data Bereich weit verbreitet ist um Datenströme zu verarbeiten.
Daher macht es Sinn, dieses in das Data-Lake-System zu integrieren und darin bereit zu stellen.
Um das System dabei nicht unnötig komplex und zu groß werden zu lassen, wird daher auf einen anderen Message-Broker verzichtet.

Da Kafka ein Event-Streaming-System ist, wird ab hier nicht mehr vom Austausch von Nachrichten, sondern von Events gesprochen.
Für diese Events müssen Topics zur Einordnung festgelegt werden.
Die Schlüssel sollten dabei so gewählt werden, dass Kafka auch für andere Datenströme verwendet werden kann, ohne, dass zu Konflikten kommt.
Daher werden die Topics der internen Kommunikation des Data-Lake-Systems immer mit "`dls\_\_"' als Prefix benannt werden.
Danach folgt der Bereich, den das Event betrifft, hier zum Beispiel "`ingestion"'.
An diesen Namen kann dann noch weiter Unterscheidung angehängt werden.
Für das Ausführen einer Ingestion wäre damit die Topic "`dls\_\_ingestion\_\_run"'.

Wenn mehrere Consumer in einer Gruppe für eine Topic sind, werden Events nicht an alle sondern immer nur an einen aus der Gruppe gesendet.
Dieser Mechanismus kann für den Lastausgleich an bestimmten Stellen verwendet werden.
Für die Ingestion kann so der Ingestion-Services einfach repliziert werden.