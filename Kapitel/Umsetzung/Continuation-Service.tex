\section{Continuation-Service}

Der Continuation-Service ist ein einzelner Prozess.
In einer Schleife werden erst alle Datenquellen überprüft, die eine Zeitsteuerung enthalten.
Hier wird eine Ingestion gestartet, wenn der Status des letzten IngestionEvents "`STOPPED"' oder "`FINISHED"' ist.
Im zweiten Schritt werden alle Datenquellen mit Datenströmen geprüft.
Datenströme werden nur bei einem Status von "`FINISHED"' automatisch gestartet, da "`STOPPED"' bedeutet, dass die Ausführung beabsichtigt beendet wurde.
Dabei wird die Schleife maximal einmal jede Minute durchlaufen, da über die Zeitsteuerung keine schneller Ausführung möglich ist.
In \cref{algo:ci-loop} ist die Logik eines Durchlaufes zu sehen.

\begin{algorithm}
    \caption{Continuation loop}
    \label{algo:ci-loop}

    loopStart $\gets$ now \

    definitions $\gets$ query DatasourceDefinitions with continuation timers \

    \ForAll {definition in definitions} {

        timers $\gets$ timers of source \

        \If {state is STOPPED or FINISHED} {

            \ForAll {timer in timers} {

                \If {timer is now} {

                    send run event \

                    end check for this DatasourceDefinition \
                }
            }
        }
    }

    definitions $\gets$ query DatasourceDefinitions with type STREAM \
    \ForAll {definition in definitions} {

        \If {state is FINISHED} {

            send run event \
        }
    }

    looEnd $\gets$ now \

    loopDuration $\gets$ loopEnd - loopStart \

    \If {loopDuration $<$ 60 sec} {

        sleep for 60 sec - loopDuration \
    }

\end{algorithm}

