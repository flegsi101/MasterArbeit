\section{Continuation-Service}

Der Continuation-Service führt durchgehend eine Schleife zum Überprüfen der Datenquellen aus.
Bei der Prüfung wird immer der Status des aktuellsten IngestionEvent betrachtet.
Je nach Datenquelle wird dann entschieden, ob eine Ingestion gestartet werden soll.
Die Logik der Schleife besteht aus drei Teilen, wovon zwei die Datenquellen prüfen und einer die Zeit steuert, nachdem die Schleife erneut ausgeführt wird (\cref{algo:ci-loop}).
Das soll nur maximal jede Minute geschehen, da auch keine schnellere Ausführung über die Timer definiert werden kann.

Im ersten Teil werden alle Datenquellen einer Datenstrom-Ingestion kontrolliert.
Für diese wird eine neue Ingestion gestartet, wenn der Status "`FINISHED"' ist.
Der Status "`STOPPED"' bedeutet, dass die Ausführung mit Absicht beendet wurde und manuell gestartet werden muss.

Der zweite Teil überprüft alle Datenquellen, bei denen ein oder mehrere Timer gesetzt wurden.
Wenn der Status des letzten IngestionEvents nicht "`STOPPED"' oder "`FINISHED"' ist, wird die Überprüfung dieser Quelle abgebrochen.
Ansonsten wird jeder Timer mit dem aktuellen Zeitpunkt verglichen.
Hier wird die Methode $is\_now$ verwendet.
Sie wird durch eine Python-Bibliothek bereitgestellt und prüft, ob ein Cron-Timer, die auch für die kontinuierliche Ingestion eingesetzt werden, zum aktuellen Zeitpunkt zutrifft.
Trifft der Timer zu, wird eine Ingestion für die Datenquelle gestartet und sofort die nächste Datenquelle überprüft.

Der dritte Teil kontrolliert die Ausführungshäufigkeit.
Dazu wird am Anfang der Schleife der Startzeitpunkt gespeichert.
Nachdem alle Überprüfungen beendet wurden, wird auch der Endzeitpunkt gespeichert.
Wenn die Differenz der beiden geringer ist als eine Minute, wird die Schleife für die Dauer der Differenz angehalten.
Damit ist sichergestellt, dass sie nicht häufiger als einmal pro Minute ausgeführt wird.

\begin{algorithm}
    \caption{Continuation loop}
    \label{algo:ci-loop}

    \KwData{-}
    \KwResult{-}

    $loopStart \gets$ current\_time() \;

    $streams \gets$ all\_datasource\_definition\_of\_type\_stream() \;
    \ForAll {$definition$ in $streams$} {
        $event \gets definition.last\_ingestion$ \;
        \If {$event.state$ is FINISHED} {
            publish\_run\_event\_for($definition.id$) \;
        }
    }

    $timed \gets$ all\_datasource\_definitions\_with\_timers() \;
    \ForAll {$definition$ in $timed$} {
        $event \gets definition.last\_ingestion$ \;
        $timers \gets definition.revision.continuation\_timers$  \;
        \If {$event.state$ is STOPPED or FINISHED} {
            \ForAll {$timer$ in $timers$} {
                \If {is\_now($timer$)} {
                    publish\_run\_event\_for($definition.id$) \;
                    \textbf{break}
                }
            }
        }
    }

    $loopEnd \gets$ current\_time() \;
    $loopDuration \gets loopEnd - loopStart$ \;
    \If {$loopDuration < 60$} {
        sleep($60 - loopDuration$) \;
    }

\end{algorithm}

