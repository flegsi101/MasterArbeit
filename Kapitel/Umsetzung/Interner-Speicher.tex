\section{Interner Datenspeicher}

Die Entwicklung einer eigenen Lösung zum Speichern von Daten mit Versionierung passt nicht in den zeitlichen Rahmen dieser Arbeit.
Daher wird der Delta Lake verwendet, da dieser alle benötigten Funktionen bietet und eine volle Schnittstelle zu Spark hat.
Da das System auf eigenen Server laufen soll, wird als Speicher für den Delta Lake ein HDFS Cluster verwendet.
Außerdem können auch die Quelldateien der Ingestion und die Plugins im HDFS abgelegt werden und so zentral durch alle Microservices erreicht werden.
Der Zugriff zum Speichern der Daten geschieht über den Delta Lake beziehungsweise Spark und für die Quelldateien und Plugins über die WebHDFS REST-Schnittstelle.

Wenn eine Datenquelle entweder mit Versionierung gespeichert wird oder Quelldateien oder Pluigns enthält, wird für diese ein Ornder im HDFS angelegt.
Diese enthält dann einen Unterordner für die Delta Tabelle und für jede Revision.
In den Revisionsordnern werden dann zugehörige Quelldatein und Plugins abgelegt. 