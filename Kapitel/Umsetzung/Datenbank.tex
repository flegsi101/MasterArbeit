\section{Metadaten-Management-System}

Da das Metadatenmodell einer Datenquelle eine verschachtelte Struktur hat, bietet sich hier als einfachste Lösung die Verwendung einer dokumenten-orientierten NoSQL-Datenbank an.
Das hat den Vorteil, dass die Sammlungen von Revisionen und IngestionEvents direkt in den Objekten der DatasourceDefinition abgelegt werden können.
In relationalen Datenbank-Systemen müsste man für jedes Modell eine eigene Tabelle erstellen und die Verknüpfungen über mehrere Abfragen zusammenführen.
Bei jeder Abfrage einer Datenquelle werden die verknüpften Einträge der IngestionEvents oder Revisionen gebraucht, was somit zu einem größeren Aufwand führt.
Außerdem gibt es viele Anfragen auf die Datenquellen, da diese nicht zwischen den Mircoservices ausgetauscht werden und so nicht im Speicher vom Service verwaltet werden können.
Daher ist es effizienter die relevanten Daten direkt mit einer Abfrage laden zu können.
Hier kann auch wieder eine Komponente aus dem Prototyp übernommen werden.
Deswegen kommt MongoDB\footnote{https://www.mongodb.com/} als Datenbank-System zum Einsatz.
