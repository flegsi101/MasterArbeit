\chapter{Einleitung}

\section{Die Motivation}
In der heutigen Zeit spielen Daten in der Welt eine immer größere Rolle.
In \textit{Rethink Data Report 2020} \textcite{rethink_data_2020} wurde eine Studie durchgeführt, die eine Steigerung von 42\% der Menge an anfallenden Daten pro Jahr prognostiziert.
Dies wird unter anderem auf den vermehrten Einsatz von IoT-Geräten, immer ausführlicheres Analysieren von Daten und die einfacher werdende Anwendung von Cloud-Speichern zurückgeführt.
Dabei besteht die Herausforderung, Daten in verschiedensten Formaten in großen Mengen zu verwalten und zu verwenden.

Ein Lösungansatz für dieses Problem sind Data-Lake-Systeme.
Data-Lake-Systeme sind zentrale Datenspeicher, die strukturierte, semi- und unstrukturierte Daten in ihrem Rohformat speichern.
Mit Hilfe von Metadaten bietet ein Datalake Schnittstellen zur Datenanalyse und -abfrage.
Dabei funktioniert das System nach dem Schema-On-Read oder auch ELT (Extrahieren Laden Transformieren) Prinzip.
Das bedeutet, dass die Daten wie bereits erwähnt im Rohformat im Data Lake gespeichert werden und erst nach dem Laden ein entsprechendes Schema angewendet wird.

Es gibt bereits viele Anbieter, die fertige Data-Lake-Systeme anbieten.
Dabei ist jedoch ein Nachteil, dass sie häufig nur in der (Cloud-)Infrastruktur des Anbieters (z.B. Microsoft Azure\footnote{https://azure.microsoft.com/}, Amazon Web Services\footnote{https://aws.amazon.com/}) verfügbar sind und sich ihrere Architektur nach diesen Diensten richtet.
Daher wird in dieser Arbeit eine Schnittstelle für die Ingestion, also das Laden der Daten in das Data-Lake-System, entwickelt, die in einem platform unabhängigen Data-Lake-System Anwendung finden soll.

\section{Der Aufbau}
Am Anfang wird auf die Ziele eigengangen, die Schnittstelle erreichen soll.
Aus diesen Zielen werden dann konkrete Anforderungen an die Entwicklung abgeleitet.
Im zweiten Kapitel wird das System der Schnittstelle entwickelt, ohne dabei auf konkrete Details, wie zum Beispiel Programmiersprachen, einzugehen.
Hier geht es mehr um die Architektur und das Design, das benötigt wird um alle Anforderunegn abzudecken.
Danach wird die Umsetzung beschrieben.
Hierbei spielt vorallem das bereits exisitierende Data-Lake-System, in dem die Ingestion-Schnittstelle integriert werden soll, eine große Rolle.
Zum Schluss wird die Ingestion-Schnittstelle evaluiert und ein Ausblick auf mögliche weitere Arbeiten gegeben.

\section{Das Existierende System}
In dem Masterprojekt \textit{Development of a Data Lake System} \parencite{datalake_proj} an der Hochschule Niederrhein wurde bereits eine Data-Lake-System-Prototyp entwickelt.
Das System ist eine monolithische Client-Server-Anwendung.
Es besteht aus einer REST-API, die zur Interaktion mit dem Data-Lake-System verwendet wird und einem Web-Frontend.
Außerdem können durch einfach Anpassungen der Server-Anwendung beliebige Datenspeicher in das Data-Lake-System integriert werden.

Als Basis wird \textit{Apache Spark}\footnote{https://spark.apache.org/} verwendet.
\textit{Apache Spark} ist eine Plattform, um Analyse auf großen Datenmengen aus zu führen.
Außerdem gibt es Schnittstellen für \textit{Scala, Java und Python} und es wird auch die Verarbeitung von Datenströmen und maschinelles Lernen unterstützt \parencite{spark}.

Der Server ist in \textit{Python} geschrieben und verwendet das Framework \textit{Flask}\footnote{https://palletsprojects.com/p/flask/} um eine REST-API bereitzustellen, über die mit dem Data Lake interagiert werden kann.
Dabei werden über die API \textit{JSON}-Objekte ausgetauscht, so dass die API client-unabhängig verwendet werden kann.
Der Client des Projekts ist eine Webanwendung, die mit \textit{Angular}\footnote{https://angular.io/} umgesetzt wurde.

\section{Verwandte Arbeiten}

\subsection{Techniken}

\subsubsection{Apache Spark}

\subsubsection{Apache Kafka}

\textit{Apache Kafka} ist ein verteiltes Event-Streaming-System, dass nach dem Publish-Subscribe-Muster funktioniert.
Events können von Produzenten in das System veröffentlicht werden und Konsumenten können diese Events abonnieren.
Das ganze läuft dabei in Echtzeit ab.
Durch seine Verteilung kann \textit{Kafka} den Ausfall einzelner Server ausgleichen.
Außerdem können Ströme von Events für einen beliebigen Zeitraum abgespeichert werden.

\textit{Kafka} besteht aus einem Cluster von Servern und verschiedenen Clients.
Es gibt zwei Arten von Servern.
Einige bilde die Speicherebene von \textit{Kafka} und werden Broaker genannt.
Die anderen verwenden \textbf{Kafka Connect}\footnote{https://kafka.apache.org/documentation/\#connect} um existierende Systeme, zum Beispiel eine Datenbank, in das Kafka Cluster zu integrieren.
Anwendungen, die entweder Events produzieren oder konsumieren sind die Clients.

In diesem System repräsentiert ein Event den Fakt, dass etwas "`passiert"' ist und besteht aus einem Schlüssel, einem Wert, einem Zeitstempel und optionalen Metadaten.
Dabei werden die Werte nicht interpretiert sonder einfach als Block versendet und können so beliebige Struktur haben.
Events werden in sogenannte Topics unterteilt.
Es kann immer mehrere Produzenten oder Konsumenten auf einer Topic geben.
Events in einer Topic können mehrfach gelesen werden und werden nicht nach dem Konsumieren gelöscht.
Es kann aber für jede Topic einzeln eine Dauer festgelegt werden, nach der die Events verworfen werden.
Um eine Topic fehlertolerant zu machen, kann diese repliziert werden.

Topics werden in Partitionen über verschiedene Broaker aufgeteilt, so dass das ganze System gut skalierbar wird.
Ein Produzenten kann zum Beispiel Events auf mehreren Brokern gleichzeitig veröffentlichen.
Wenn ein Event in einer Topic veröffentlicht wird, wird dieses an eine der Partitionen angehängt.
Events, die den gleichen Schlüssel haben werden immer der gleichen Partition zugeordnet und Events einer Partition kommen garantiert in der Reihenfolge des Schreibens bei dem Konsumenten der Partition an \parencite{kafka-docs}.

\textit{Apache Kafka} wird im Big Data Bereich weit verbreitet um Datenströme zu verarbeiten.
Daher macht es Sinn, \textit{Kafka} auch in dieses Data Lake System zu integrieren und darin bereit zu stellen.
Außerdem kann es auch für die Kommunikation zwischen den verschiedenen Microservices verwendet werden.

\subsection{Hadoop Distributed File System}

Als Speicher wird das \textit{Hadoop Distributed File System (HDFS)} verwendet.
Das \textit{HDFS} ist ein verteiltes, auf große Dateien ausgelegtes Dateisystem.
Ein \textit{HDFS} Cluster besteht aus einem Namenode und mehreren Datanodes.

Der Namenode verwaltet den Baum des Dateisystems und kontrolliert den Zugriff durch Clients.
Zusätzlich führt er Operation auf dem Dateisystem aus.
Dazu zählen das Öffnen, Schließen oder Umbenennen von Dateien oder Ordnern.
Die Dateien selbst werden in Blöcke aufgeteilt auf den Datanodes gespeichert.
Diese sind auch dafür verantwortlich, Lese- und Schreibanfragen zu bedienen und verwalten die Erstellung, Löschung und Replikation unter Anleitung des Namenodes \parencite{hdfs}.

Das \textit{HDFS} ist ebenfalls eine weit verbreitete Technik im Big Data Bereich und eignet sich auch hier, durch die Auslegung auf große Dateien, sehr gut um die Quelldateien der Datenquellen abzulegen.
Außerdem sind Dateien auf allen Servern verfügbar, da es sowohl eine REST-Schnittstelle als auch Unterstützung in \textit{Apache Spark} gibt.
