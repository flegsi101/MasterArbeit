\chapter{Einleitung}

Heutzutage spielen Daten und deren Verarbeitung in vielen Bereichen eine immer wichtigere Rolle.
In \textit{Rethink Data Report 2020} \textcite{rethink_data_2020} wurde eine Studie durchgeführt, die eine Steigerung von 42\% der Menge an anfallenden Daten pro Jahr prognostiziert.
Dies wird unter anderem auf den vermehrten Einsatz von IoT-Geräten, immer ausführlicheres Analysieren von Daten und die einfacher werdende Anwendung von Cloud-Speichern zurückgeführt.
Dabei besteht die Herausforderung, Daten in verschiedensten Formaten in großen Mengen zu verwalten und zu verwenden.

Ein Ansatz dafür sind Data-Lake-Systeme.
Das sind Systeme, die Daten in ihrem Rohformat speichern und erst nach dem Laden für die Verwendung transformieren. Ein ideales Data-Lake-System sollte mehrerer Anforderungen erfüllen.
Darunter zählt die Unterstützung jeglicher Datenquellen, wie zum Beispiel Datenbanken, Dateien, Datenströmen oder auch andere Schnittstellen wie REST-APIs.
Da sich Daten mit der Zeit ändern können sollte es auch möglich sein, den Verlauf der Daten, als Versionen, zu speichern und abfragen zu können.
Eine Metadatenverwaltung kann die Verwaltung und das Abfragen von Daten einfacher und klarer machen und sollte deshalb ebenfalls ein Bestandteil sein.
Da Datenauswertungen und -analysen nicht mehr nur durch Menschen, sondern auch durch Methoden der künstlichen Intelligenz ausgeführt werden, sollte der Data Lake eine Anbindung an diese und eine entsprechende Aufbereitung der Daten bieten.

Es gibt bereits viele Anbieter, die fertige Data-Lake-Systeme anbieten.
Dabei ist jedoch ein Nachteil, dass sie häufig nur in der (Cloud-)Infrastruktur des Anbieters (z.B. Microsoft Azure\footnote{https://azure.microsoft.com/}, Amazon Web Services\footnote{https://aws.amazon.com/}) verfügbar sind und sich ihrere Architektur nach diesen Diensten richtet.

In einem Masterprojekt \citetitle{datalake_proj}\parencite{datalake_proj} wurde ein Prototyp eines Data-Lake-Systems entwickelt.
In dieser Arbeit wird eine Schnittstelle für die Ingestion, also das Laden der Daten in das Data-Lake-System, entwickelt, die in diesen Prototyp integriert werden soll.

\subsection{Wie sieht ein Data Lake aus}


In  wird ein Data-Lake-System dargestellt, das alle in der Einleitung angesprochenen Punkte berücksichtigt.
Hier ist nochmal zu sehen, dass Daten sowohl über eine einmalige als auch eine kontinuierliche Ingestion in den Data Lake gelangen können.
Dabei verwenden beide Komponenten eine einheitliche und allgemeine Definition (Ingestion-Flow-Definition) über die von einer einfachen Datenbank als Quelle bis hin zu einem komplexen Ablauf von Abfragen gegen eine API festgelegt werden können.
Das Ziel ist es, die Ingestion-Schnittstelle komplett unabhängig von der Datenquelle zu machen.
In der Continuous-Integration ist zusätzlich zu sehen, dass hier neben dem reinen Laden der Daten auch das Finden und Speichern von Deltas eine Rolle spielt, da hier durch das kontinuierliche Laden der Daten Änderungen in den Daten anfallen.

Die Daten des Data-Lake-Systems können dann für maschinelles Lernen aufbereitet und verwendet werden.
Ergebnisse des maschinelles Lernen werden wieder im Data Lake gespeichert.
Die Herausforderung dabei ist, die Ergebnisse mit in den Versionsverlauf einzupflegen.
Eine Möglichkeit wäre das Machine-Learning als Datenquelle an die Ingestion mit anzubinden.
Um diese Schwierigkeit zu umgehen, kann man die Informationen auch direkt in den Data Lake zurückschreiben.
Dabei verliert man aber alle Vorteile des Versionsverlauf.

Für die Benutzer des Data-Lake-Systems stehen drei Interaktionspunkte zur Verfügung.
Der erste ist die Erstellung und Bearbeitung von Ingestion-Definitions.
Diese beschreiben, wie Daten aus verschiedenen Quellen geladen werden sollen und wie man in den Datenquellen Änderungen findet.
Um das System benutzerfreundlicher zu gestalten, sollten bereits Definitionen für gängige Systeme vorhanden sein, die dann erweitert oder als Vorlage genutzt werden können.
Als zweites steht eine Abfragen-Schnittstelle zur Verfügung, die einen einheitlichen Zugriff auf die verschieden Abfragemöglichkeiten des Systems stellt.
Zum Schluss gibt es noch die Metadatenverwaltung, bei der zu den automatisch erstellten Metadaten weitere hinzugefügt werden können.

\section{Der Aufbau}
Am Anfang wird auf die Ziele eigengangen, die Schnittstelle erreichen soll.
Aus diesen Zielen werden dann konkrete Anforderungen an die Entwicklung abgeleitet.
Im zweiten Kapitel wird das System der Schnittstelle entwickelt, ohne dabei auf konkrete Details, wie zum Beispiel Programmiersprachen, einzugehen.
Hier geht es mehr um die Architektur und das Design, das benötigt wird um alle Anforderunegn abzudecken.
Danach wird die Umsetzung beschrieben.
Hierbei spielt vorallem das bereits exisitierende Data-Lake-System, in dem die Ingestion-Schnittstelle integriert werden soll, eine große Rolle.
Zum Schluss wird die Ingestion-Schnittstelle evaluiert und ein Ausblick auf mögliche weitere Arbeiten gegeben.

\section{Verwandte Arbeiten}

\subsection{Change Data Capture (CDC)}

Unter Change Data Capture versteht man das Erfassen von Änderungen an Daten in Datenspeichern.
Dies wird gemacht um die erfassten Änderungen an andere Systeme weitergeben zu können, die die Änderungen dann weiter auswerten.
Dieses Vorgehen spielt mittlerweile in viele Bereichen eine Rolle und es gibt einige Arbeiten, die sich mit dem Thema und verschiedenen Vorgehen dazu befassen.
In den Arbeiten \citetitle{delta-view_gen}\parencite{delta-view_gen}, \citetitle{cdc_in_nosql}\parencite{cdc_in_nosql} und \citetitle{boeing}\parencite{boeing} werden vier Ansätze für die Umsetzung von CDC erläutert.

\subsubsection{Snapshot basiert}
Der erste Ansatz ist das Vergleichen von zwei Momentaufnahmen (Snapshots) eines Datensatzes.
Dabei wird bei jedem Durchlauf der CDC zuerst ein aktueller Snapshot generiert.
Danach wird dieser Snapshot mit dem Snapshot des vorherigen Durchlaufs verglichen, um alle Änderungen zu erhalten.
Hierfür muss ein separater Speicherort für diese Snapshots festgelegt werde.
Ein Nachteil dieser Methode ist, dass man nicht den gesamten Änderungsverlauf zwischen zwei Snapshots nachvollziehen kann.
Außerdem müssen den Vergleich immer alle Daten geladen werden, was zu einem hohen Rechen- und Speicheraufwand führen kann \cite{cdc_in_nosql}.

Zwei Ansätze eine Liste von Einfügungen, Änderungen und Löschungen aus zwei Snapshots zu erhalten, die auch von \citeauthor{cdc_in_nosql} \cite{cdc_in_nosql} referenziert werden, haben W. Labio und H. Garcia-Molina \cite{snapshot_algos} dargelegt.
Dazu werden alle Daten als Einträge mit einem einzigartigen Schlüssel und den dazu gehörigen Daten betrachtet.

Der erst genannte Ansatz basiert auf Join-Algorithmen, wie sie in der Informatik, zum Beispiel in \cite{joins}, schon viel besprochen wurden.
Wenn man die Einträge beider Snapshots über ihren Schlüssel verknüpft, kann man so durch einen Vergleich der Ergebnisse alle geänderten Einträge finden.
Um dann noch alle Einfügungen und Löschungen zu finden, kann man einen Outerjoin durchführen.
Bei einem Outerjoin erhält man alle Daten, die nur in einem der beiden Datensätze auftauchen.
Je nachdem in welchem Datensatz die Daten auftauchen, handelt es sich dann um eine Einfügung oder Löschung.

Als zweites wurde ein neuerer Algorithmus präsentiert, der mit der Annahme arbeitet, dass die Einträge in beiden Snapshots nah beieinander liegen.
Bei diesem Algorithmus wird ein Fenster von fester Länge über beide Snapshots geschoben und nur Einträge aus diesem Fenster verglichen.
Bei diesem Vorgehen reicht es aus, beide Snapshots nur einmal zu lesen.
Es kann aber je nach Aufteilung der Einträge passieren, dass sogenannte unnütze Einträge entstehen.
Diese werden definiert als eine Folge von Änderungen, die keine Auswirkung auf das Endergebnis hat.
Das sind Folgen, bei denen erst ein Eintrag gelöscht und dann eingefügt wird, oder andersherum.

\subsubsection{Zeitstempel basiert}
Ein weiterer Ansatz ist die Verwendung von Zeitstempeln.
Diese enthalten immer den Zeitpunkt der letzen Änderung.
Um Zeitstempel für die CDC zu verwenden werden, muss jedem Dateneintrag eine weitere Spalte (der Begriff Spalte bezieht sich hier nicht nur auf tabellen-artige Daten, sondern meint ein Feld, dass jeder Datensatz enthält) dafür hinzugefügt werden.
Auf diese Art kann man ebenfalls nur die kumulierten Änderungen seit dem letzten Durchlauf und keine Löschungen erfassen.

Ohne weiteren Aufwand ist es bei diesem Vorgehen auch nicht möglich zu unterscheiden, ob ein Datensatz sich geändert hat oder neu eingefügt wurde, da der Zeitstempel darüber keine Aussage treffen kann \cite{delta-view_gen}.
In \cite{cdc_in_nosql} wird als Lösung dafür vorgeschlagen, zwei Zeitstempel zu verwenden. Einer enthält den Zeitpunkt der Erstellung und der andere den der letzten Änderung. Diese werden dann mit dem Zeitpunkt des letzten CDC Durchlaufs verglichen. Wenn beide hinter dem letzten Durchlauf liegen, handelt es sich um eine Einfügung und wenn der erste vor und der zweite dahinter liegen um eine Aktualisierung.

Ein weiteres Problem kann sein, dass auch der Aufwand zum Finden der relevanten Zeitstempel bei vielen Daten hoch sein kann. Am Beispiel von SQL-Datenbanken bedeutet das, wenn es keinen Index auf die Zeitstempelspalten gibt, muss die gesamte Tabelle gelesen werden \cite{boeing}.

\subsubsection{Trigger basiert}
Trigger sind Programm-Codes, die vom Datenbanksystem bei verschiedenen Events ausgeführt werden.
Diese können verwendet werden, um alle Änderungen zu erfassen und für das CDC-Programm festzuhalten.
Die Grundvoraussetzung dafür ist, dass das System, auf dem die Daten verwaltet werden diese auch unterstützt.
Mit Hilfe der Trigger können alle Änderungen, auch die, die zwischen zwei CDC-Durchläufen passieren, erfasst werden.

\subsubsection{Log basiert}
Die letzte Möglichkeit ist, die Logfunktionen eines Systems zu nutzen.
Die meisten Datenbanksysteme zum Beispiel benutzen selbst Logs, um Wiederherstellungen möglich zu machen \cite{delta-view_gen}.
Diese können dann auch von dem CDC-Programm ausgelesen und verwendet werden.
Hierdurch gibt es fast keinen extra Aufwand für das eigentliche System.
Aber auch hier gilt, dass diese Art nur bei Systemen verwendet werden kann, die eine entsprechende Logfunktion bieten.
Ein weiterer Nachteil ist, dass für verschiedene Systeme herstellerspeziefische Implementierungen benötigt werden \cite{delta-view_gen}.

\subsubsection{Anwendbarkeit}

Im Hinblick auf diese Arbeit kann festgehalten werden, dass die meisten Methoden zur CDC direkt bei den Datenspeichern ausgeführt werden müssen, in denen die Daten geändert werden.
Nur das Vorgehen, bei denen zwei Momentaufnahmen von Daten verglichen werden kann ortsunabhänigig ausgeführt werden.
Daher ist dieses Vorgehen auch die bessere Wahl, wenn es darum geht Datenquellen unabhängig die Möglichkeit zu geben Änderungsdaten zu berechnen.
Da dieses Vorgehen aber weniger effizient ist, sollte eine Ingestion-Schnittstelle sowohl die Möglichkeit bieten diese Änderungsdaten im Data Lake System zu erzeugen als auch von externen CDC-System Daten einzuspielen.

\section{Techniken}

\subsection{Apache Spark}

\textit{Apache Spark} ist eine einheitliche Engine für Verarbeitung von verteilten Daten mit verschieden Arbeitsabläufen in einem Cluster.
Es gibt Bibliotheken, die beispielsweise Arbeiten mit SQL, Datenströmen, maschinelles Lernen oder Graphen ermöglicht.
Diese Bibliotheken laufen dafür auf dieser gemeinsamen Engine.
Durch Optimierungen der Implementationen erreicht Spark eine ähnliche Performance, wie spezialisierte Engines.

Ein Kernpunkt in Spark ist die Abstraktion in RDDs (Resilient Distributed Datasets, deutsch: Robuste Verteilte Datensätze).
RDDs sind fehlertolerante Sammlungen von Objekten, die auf dem Cluster verteilt sind und parallel bearbeitet werden können.
Diese werden flüchtig im Speicher gehalten, können aber für einen schnelleren Zugriff zwischengespeichert werden.
Die Erstellung und Bearbeitung von RDDs geschieht über sogenannte Transformationen.
Die Transformationen werden in einem Herkunftsgraphen gespeichert, wodurch eine Wiederherstellung bei Fehler an jedem Punkt möglich ist.

Für die Verarbeitung von strukturierten oder semi-strukturierten, tabellen-artigen Daten gibt es zusätzlich SparkSQL.
Spark biete APIs für die Sprachen Scala, Java, Python und R.
Auf den RDDs gibt es noch eine weitere Abstraktionsebene.
Mit DataFrames, die eine Sammlung RDDs von Datensätzen mit einem bekannten Schema sind, kann eine API benutzt werden, bei der die Bearbeitung der Daten über Funktionsaufrufe statt SparkSQL möglich ist \parencite{spark}. 

\subsubsection{Arbeiten mit Spark}

Die Interaktion mit einem Spark Cluster kann über eine interaktive Shell oder als fertige Anwendung geschehen.
Im ersten Schritt wird eine SparkSession erzeugt.
Diese enthält verschiedene Informationen zu Anwendung.
Als SparkMaster kann man sowohl ein Cluster verwenden als auch den lokalen Computer.
So ist es möglich eine mit Spark zu Arbeiten ohne eine Cluster starten zu müssen.
Um fehlende Bibliotheken für die Verarbeitung nach zu laden, kann der SparkSession eine Liste mit Maven-Abhänigkeiten mitgegeben werden.
Außerdem wird in der SparkSession der Name der Anwendung gesetzt.
Dieser Name wird dann auch in der Cluster UI angezeigt.

Nachdem eine Session erstellt ist, können DataFrames erstellt werden.
Hier ist es möglich eine leeres zu erzeugen oder direkt über Spark Daten aus einer Datenquelle in ein Dataframe zu laden.
Dazu werden die DataFrame- und DataStreamReader verwendet.
In den Readern werden sowohl das Format der zu lesenden Daten sowie formatabhängige Optionen und eventuell auch das Schema der Daten festgelegt.
Optionen können zum Beispiel Verbindungsinformationen bei einer Datenbank oder der Pfad bei Dateien sein.

Die geladenen DataFrames können dann bearbeitet oder gespeichert werden.
Das Speichern geschieht über DataFrame- und DataStreamWriter, die analog zu den Readern verwendet werden \parencite{spark-website}.

\subsection{Apache Kafka}

\textit{Apache Kafka} ist ein verteiltes Event-Streaming-System, dass nach dem Publish-Subscribe-Muster funktioniert.
Events können von Produzenten in das System veröffentlicht werden und Konsumenten können diese Events abonnieren.
Das ganze läuft dabei in Echtzeit ab.
Durch seine Verteilung kann \textit{Kafka} den Ausfall einzelner Server ausgleichen.
Außerdem können Ströme von Events für einen beliebigen Zeitraum abgespeichert werden.

\textit{Kafka} besteht aus einem Cluster von Servern und verschiedenen Clients.
Es gibt zwei Arten von Servern.
Einige bilde die Speicherebene von \textit{Kafka} und werden Broaker genannt.
Die anderen verwenden \textbf{Kafka Connect}\footnote{https://kafka.apache.org/documentation/\#connect} um existierende Systeme, zum Beispiel eine Datenbank, in das Kafka Cluster zu integrieren.
Anwendungen, die entweder Events produzieren oder konsumieren sind die Clients.

In diesem System repräsentiert ein Event den Fakt, dass etwas "`passiert"' ist und besteht aus einem Schlüssel, einem Wert, einem Zeitstempel und optionalen Metadaten.
Dabei werden die Werte nicht interpretiert sonder einfach als Block versendet und können so beliebige Struktur haben.
Events werden in sogenannte Topics unterteilt.
Es kann immer mehrere Produzenten oder Konsumenten auf einer Topic geben.
Events in einer Topic können mehrfach gelesen werden und werden nicht nach dem Konsumieren gelöscht.
Es kann aber für jede Topic einzeln eine Dauer festgelegt werden, nach der die Events verworfen werden.
Um eine Topic fehlertolerant zu machen, kann diese repliziert werden.

Topics werden in Partitionen über verschiedene Broaker aufgeteilt, so dass das ganze System gut skalierbar wird.
Ein Produzenten kann zum Beispiel Events auf mehreren Brokern gleichzeitig veröffentlichen.
Wenn ein Event in einer Topic veröffentlicht wird, wird dieses an eine der Partitionen angehängt.
Events, die den gleichen Schlüssel haben werden immer der gleichen Partition zugeordnet und Events einer Partition kommen garantiert in der Reihenfolge des Schreibens bei dem Konsumenten der Partition an \parencite{kafka-docs}.
\subsection{Hadoop Distributed File System}

Das \textit{Hadoop Distributed File System (HDFS)} ist ein verteiltes und fehlertolerantes Dateisystem.
Es wurde designt um auf Hardware mit geringen Kosten zu laufen und große Datenmengen zu verarbeiten.
Im HDFS gespeicherte Dateien können von einem Gigabyte bis mehrere Terabyte groß sein.
Außerdem unterstützt HDFS die replizierte Speicherung von Dateien mit einem beliebigen Faktor.

Das Dateisystem ist ähnlich zu anderen bekannten Dateisystem.
Dateien und Ordner können im Namensraum hierarchisch organisiert werden.
Es unterstützt jedoch keine Zurgiffsberechtigung oder Hard- und Soft-Links.
Um einfach und effektiv kohärent zu bleiben, werden Dateien nur einmal geschrieben, können aber mehrfach gelesen werden.
Dateien werden zur Speicherung in einzelne Blöcke aufgeteilt.
Dabei sind für eine Datei alle, bis auf der letzte, Blöcke gleich groß.

Ein HDFS-Cluster funktioniert nach dem Master-Worker-Prinzip und besteht aus einem NameNode und vielen DataNodes.
Der NameNode übernimmt die Verwaltung des Namensraum und Verteilung der einzelnen Blöcke einer Datei.
Er reguliert dazu noch den Zugriff durch Clients und führt Operationen auf dem Dateisystem, wie das Öffnen, Schließen oder Umbenennen von Ordnern und Dateien aus.
Die DataNodes speichern die einzelnen Blöcke der Dateien.
Auf die Anweisung des NameNodes führen sie die Erstellung, Löschung und Replikation von Blöcken aus.
Außerdem bearbeiten sie Anfragen zum Lesen und Schreiben von Dateien \parencite{hdfs}.
\subsection{Delta Lake}

Delta Lake ist eine extra Speicher-Ebene, die auf Objektspeichern in der Cloud (z.B. Amazon S3) oder anderen Dateisystemen verwendet werden kann.
Das Ziel ist es diesen Speichern ACID Transaktionen, schnelles Arbeiten mit Metadaten der Tabelle und eine Versionierung der Daten hinzuzufügen.
Daten werden in sogenannten Delta Tabellen mit Metadaten und Logs gespeichert.

Eine Delta Tabelle ist praktisch ein Verzeichnis im Speicher.
Die tatsächlichen Daten werden hier in \textit{Apache Parquet} Dateien abgelegt.
Parquet ist ein spalten-orientiertes Format, zum effizienten Speichern von Daten \parencite{parquet}.
Dabei können die Daten auch noch in Unterverzeichnisse aufgeteilt sein, zum Beispiel für jedes Datum ein Verzeichnis.
Neben den Datenverzeichnissen gibt es eines für die Logs in Form von JSON Dateien mit aufsteigender Nummerierung.
Metadaten werden sowohl innerhalb der Parquet als auch der Log Dateien passend gespeichert.

Im Delta Lake wird ein Protokoll für den Zugriff verwendet, dass es ermöglicht, dass mehrere Clients gleichzeitig Lesen und immer nur einer Schreiben kann.
Dabei werden beim Schreiben immer erst neue Datensätze, die zur Tabelle hinzugefügt werden sollen in das korrekte Verzeichnis geschrieben.
Danach wird eine neu Log Datei erstellt.

Beim Lesen werden die Log Dateien als Grundlage verwendet um daraus zusammen mit den gespeicherten Daten den Tabellen stand zu erzeugen.
Man kann beim Lesen auch eine bestimmte Version angeben.
Um den Aufwand bei der Verarbeitung der Logs zu verringern wird periodisch ein Kontrollpunkt erzeugt, bei dem alle vorherigen Logs zusammengefügt und komprimiert werden.
Das bedeutet, dass zum Beispiel das Operationen die sich gegenseitig aufheben nicht gespeichert werden.
Damit reicht es aus nur den letzten Checkpoint vor der zu lesenden Version und alle darauf folgende Logs zu lesen.

Durch das Design werden keine eignen Server für die Pflege der Delta Tabellen benötigt, sondern dies kann alles über den Client gemacht werden.
Der Delta Lake unterstützt sowohl die Batch-Verarbeitung von Daten als auch Datenströme und bietet volle Integration in Spark \parencite{deltalake}.




\section{Das Existierende System}
In dem Masterprojekt \textit{Development of a Data Lake System} \parencite{datalake_proj} an der Hochschule Niederrhein wurde bereits eine Data-Lake-System-Prototyp entwickelt.
Das System ist eine monolithische Client-Server-Anwendung.
Es besteht aus einer REST-API, die zur Interaktion mit dem Data-Lake-System verwendet wird und einem Web-Frontend.
Außerdem können durch einfach Anpassungen der Server-Anwendung beliebige Datenspeicher in das Data-Lake-System integriert werden.
Als Plattform für die Datenverarbeitung wird \textit{Apache Spark} verwendet.

Der Server ist in \textit{Python} geschrieben und verwendet das Framework \textit{Flask}\footnote{https://palletsprojects.com/p/flask/} um eine REST-API bereitzustellen, über die mit dem Data Lake interagiert werden kann.
Dabei werden über die API \textit{JSON}-Objekte ausgetauscht, so dass die API client-unabhängig verwendet werden kann.
Der Client des Projekts ist eine Webanwendung, die mit \textit{Angular}\footnote{https://angular.io/} umgesetzt wurde.

