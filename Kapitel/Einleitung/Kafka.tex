\subsection{Apache Kafka}
\label{sec:kafka}

Apache Kafka ist ein verteiltes Event-Streaming-System.
Die Vermittlung von Events nach dem Publish-Subscribe-Muster läuft dabei in Echtzeit.
Events können von Produzenten veröffentlicht werden und Konsumenten können auf diese Events abonnieren.
Durch seine Verteilung kann Kafka den Ausfall einzelner Server ausgleichen.
Zusätzlich können Ströme von Events für einen beliebigen Zeitraum persistiert werden.

Kafka besteht aus einem Cluster von Servern und verschiedenen Clients.
Es gibt zwei Arten von Servern.
Die sogenannten Broaker sind für die Verteilung und Verwaltung von Events zuständig.
Andere verwenden Kafka Connect\footnote{https://kafka.apache.org/documentation/\#connect} um existierende Systeme, zum Beispiel eine Datenbank, in das Kafka Cluster zu integrieren.
Die Clients sind Anwendungen die entweder Events produzieren oder konsumieren.

In diesem System repräsentiert ein Event den Fakt, dass etwas "`passiert"' ist und besteht aus einem Schlüssel, einem Wert, einem Zeitstempel und optionalen Metadaten.
Dabei werden die Werte nicht interpretiert sonder einfach als Block versendet und können so beliebige Struktur haben.
Events werden in sogenannte Topics unterteilt.
Es kann immer mehrere Produzenten oder Konsumenten auf einer Topic geben.
Events in einer Topic können mehrfach gelesen werden und werden nicht nach dem Konsumieren gelöscht.
Es kann aber für jede Topic einzeln eine Dauer festgelegt werden, nach der die Events verworfen werden.
Um eine Topic fehlertolerant zu machen, kann diese repliziert werden.

Topics werden in Partitionen über verschiedene Broaker aufgeteilt, so dass das ganze System gut skalierbar wird.
Ein Produzenten kann zum Beispiel Events auf mehreren Brokern gleichzeitig veröffentlichen.
Wenn ein Event in einer Topic veröffentlicht wird, wird dieses an eine der Partitionen angehängt.
Events, die den gleichen Schlüssel haben werden immer der gleichen Partition zugeordnet und Events einer Partition kommen garantiert in der Reihenfolge des Schreibens bei dem Konsumenten der Partition an \parencite{kafka-docs}.