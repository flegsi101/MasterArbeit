\subsection{Change Data Capture (CDC)}
\label{sec:cdc}

Um Änderungen an Daten in Datenquellen in das Data-Lake-System einpflegen zu können, müssen diese erst erfasst werden.
Diesen Prozess nennt man Change-Data-Capture (CDC).
Das Ziel beim CDC ist es, die Änderungen an den Daten nur an einer Stelle zu erfassen und dann an andere Systeme weiter zu schicken.
Dafür gibt es verschiedene Ansätze, die in den Arbeiten \citetitle{delta-view_gen}\parencite{delta-view_gen}, \citetitle{cdc_in_nosql}\parencite{cdc_in_nosql} und \citetitle{boeing}\parencite{boeing} erläutert werden.

\subsubsection{Datenbank-Trigger basiert}
Datenbank-Trigger sind Events, die bei verschiedenen Aktionen auf den Daten in einer Datenbank ausgelöst werden.
Über diese Trigger lassen sich CDC-Programme realisieren, die Änderungen genau dann festhalten, wenn sie geschehen.
Ein Nachteil ist, dass die Methode nur in System angewendet werden kann, die auch Trigger unterstützen.
Dafür ist es möglich alle Änderungen wie Einfügen, Aktualisieren oder Löschen von Daten zu erfassen \parencite{boeing}.

\subsubsection{Log basiert}
Es gibt viele Datenspeicher-Systeme, die Logs über die Aktionen auf den Daten führen.
Diese werden zum Beispiel genutzt um eine Wiederherstellung möglich zu machen.
Ein CDC-Programm kann diese Logs auslesen und daraus die Änderungsdaten erzeugen.
Hierdurch gibt es fast keinen extra Aufwand für das eigentliche System.
Aber auch hier gilt, dass diese Methode davon abhängig ist ob ein System Logs erstellt und ob diese durch externe Programme abgerufen werden können \parencite{delta-view_gen}.


\subsubsection{Zeitstempel basiert}
Ein weiterer Ansatz ist die Verwendung von Zeitstempeln mit den Zeitpunkten der Erstellung und letzten Änderung.
Diese Zeitstempel müssen jedem Datensatz vorhanden sein.
Die Verantwortung dafür kann entweder bei dem Ersteller der Daten liegen oder durch das Speichersystem automatisch hinzugefügt werden.
Das CDC-Programm überprüft regelmäßig alle Zeitstempel der Einträge in den Daten.
Wenn diese zwischen dem letzten und dem aktuellen Durchlauf liegen wird die Änderung erfasst.
Hierbei werden nur kumulierte Änderungen seit dem letzten Durchlauf erfasst.
Es ist nicht möglich nachzuvollziehen, welche und wie viele Änderungen in der Zeit gemacht wurden.
Außerdem lassen sich auch mit dieser Methode kein Löschungen erfassen \parencite{delta-view_gen}.
Der Aufwand für diese Methode kann relativ hoch werden, da ohne Indices auf den Zeitstempeln immer die gesamten Daten gelesen werden müssen \cite{boeing}.

\subsubsection{Snapshot basiert}
Die letzte Methode ist das Vergleichen zweier Momentaufnahmen (Snapshots) eines Datensatzes.
Dabei wird bei jedem Durchlauf zuerst ein aktueller Snapshot generiert.
Dieser wird danach mit dem des vorherigen Durchlaufs verglichen, um alle Änderungen zu erhalten.
Hierfür muss ein separater Speicherort für die Snapshots festgelegt werde.
Wie bei den Zeitstempeln ist es nicht möglich den gesamten Änderungsverlauf zwischen zwei Snapshots nachzuvollziehen.
Außerdem müssen für den Vergleich immer alle Daten geladen werden, was zu einem hohen Rechen- und Speicheraufwand führen kann \cite{cdc_in_nosql}.

% den Teil eher woanders
% Zwei Ansätze eine Liste von Einfügungen, Änderungen und Löschungen aus zwei Snapshots zu erhalten, die auch von \citeauthor{cdc_in_nosql} \cite{cdc_in_nosql} referenziert werden, haben W. Labio und H. Garcia-Molina \cite{snapshot_algos} dargelegt.
% Dazu werden alle Daten als Einträge mit einem einzigartigen Schlüssel und den dazu gehörigen Daten betrachtet.

% Der erst genannte Ansatz basiert auf Join-Algorithmen, wie sie in der Informatik, zum Beispiel in \cite{joins}, schon viel besprochen wurden.
% Wenn man die Einträge beider Snapshots über ihren Schlüssel verknüpft, kann man so durch einen Vergleich der Ergebnisse alle geänderten Einträge finden.
% Um dann noch alle Einfügungen und Löschungen zu finden, kann man einen Outerjoin durchführen.
% Bei einem Outerjoin erhält man alle Daten, die nur in einem der beiden Datensätze auftauchen.
% Je nachdem in welchem Datensatz die Daten auftauchen, handelt es sich dann um eine Einfügung oder Löschung.

% Als zweites wurde ein neuerer Algorithmus präsentiert, der mit der Annahme arbeitet, dass die Einträge in beiden Snapshots nah beieinander liegen.
% Bei diesem Algorithmus wird ein Fenster von fester Länge über beide Snapshots geschoben und nur Einträge aus diesem Fenster verglichen.
% Bei diesem Vorgehen reicht es aus, beide Snapshots nur einmal zu lesen.
% Es kann aber je nach Aufteilung der Einträge passieren, dass sogenannte unnütze Einträge entstehen.
% Diese werden definiert als eine Folge von Änderungen, die keine Auswirkung auf das Endergebnis hat.
% Das sind Folgen, bei denen erst ein Eintrag gelöscht und dann eingefügt wird, oder andersherum.

% \subsubsection{Anwendbarkeit}

% Im Hinblick auf diese Arbeit kann festgehalten werden, dass die meisten Methoden zur CDC direkt bei den Datenspeichern ausgeführt werden müssen, in denen die Daten geändert werden.
% Nur das Vorgehen, bei denen zwei Momentaufnahmen von Daten verglichen werden kann ortsunabhänigig ausgeführt werden.
% Daher ist dieses Vorgehen auch die bessere Wahl, wenn es darum geht Datenquellen unabhängig die Möglichkeit zu geben Änderungsdaten zu berechnen.
% Da dieses Vorgehen aber weniger effizient ist, sollte eine Ingestion-Schnittstelle sowohl die Möglichkeit bieten diese Änderungsdaten im Data Lake System zu erzeugen als auch von externen CDC-System Daten einzuspielen.