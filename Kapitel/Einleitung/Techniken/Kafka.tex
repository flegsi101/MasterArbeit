\subsection{Apache Kafka}

\textit{Apache Kafka} ist ein verteiltes Event-Streaming-System, dass nach dem Publish-Subscribe-Muster funktioniert.
Events können von Produzenten in das System veröffentlicht werden und Konsumenten können diese Events abonnieren.
Das ganze läuft dabei in Echtzeit ab.
Durch seine Verteilung kann \textit{Kafka} den Ausfall einzelner Server ausgleichen.
Außerdem können Ströme von Events für einen beliebigen Zeitraum abgespeichert werden.

\textit{Kafka} besteht aus einem Cluster von Servern und verschiedenen Clients.
Es gibt zwei Arten von Servern.
Einige bilde die Speicherebene von \textit{Kafka} und werden Broaker genannt.
Die anderen verwenden \textbf{Kafka Connect}\footnote{https://kafka.apache.org/documentation/\#connect} um existierende Systeme, zum Beispiel eine Datenbank, in das Kafka Cluster zu integrieren.
Anwendungen, die entweder Events produzieren oder konsumieren sind die Clients.

In diesem System repräsentiert ein Event den Fakt, dass etwas "`passiert"' ist und besteht aus einem Schlüssel, einem Wert, einem Zeitstempel und optionalen Metadaten.
Dabei werden die Werte nicht interpretiert sonder einfach als Block versendet und können so beliebige Struktur haben.
Events werden in sogenannte Topics unterteilt.
Es kann immer mehrere Produzenten oder Konsumenten auf einer Topic geben.
Events in einer Topic können mehrfach gelesen werden und werden nicht nach dem Konsumieren gelöscht.
Es kann aber für jede Topic einzeln eine Dauer festgelegt werden, nach der die Events verworfen werden.
Um eine Topic fehlertolerant zu machen, kann diese repliziert werden.

Topics werden in Partitionen über verschiedene Broaker aufgeteilt, so dass das ganze System gut skalierbar wird.
Ein Produzenten kann zum Beispiel Events auf mehreren Brokern gleichzeitig veröffentlichen.
Wenn ein Event in einer Topic veröffentlicht wird, wird dieses an eine der Partitionen angehängt.
Events, die den gleichen Schlüssel haben werden immer der gleichen Partition zugeordnet und Events einer Partition kommen garantiert in der Reihenfolge des Schreibens bei dem Konsumenten der Partition an \parencite{kafka-docs}.