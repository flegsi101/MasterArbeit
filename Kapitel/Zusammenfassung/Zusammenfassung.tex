\chapter{Zusammenfassung}

\subsection{Ergebnis}
Im Rahmen dieser Arbeit wurde eine Schnittstelle entwickelt, die es ermöglicht über die eine einheitliche Definition Daten aus allen möglich Quellen in das Data-Lake-System zu integrieren.
Die Schnittstelle kann über einen Web-Service bereitgestellt und durch beliebige Clients genutzt werden.

Um eine Abdeckung aller Datenquellen zu erreichen wird neben der Verwendung von Apache Spark (\cref{sec:spark}) auch eine ein Plugin-System verwendet.
Neben dem Laden aus Datenspeichern, können auch Dateien über die Schnittstelle in das Data-Lake-System hochgeladen werden.

Die Versionierung ist für strukturierte und semistrukturierte Daten, also zum Beispiel Tabellen oder JSON-Dateien, möglich.
Es besteht die Wahl zwischen der internen Berechnung von Änderungen zwischen zwei Versionen oder dem Einpflegen von Änderungsdaten aus eine eigenen Change-Data-Capture-Lösung (\cref{sec:cdc}).

Die Architektur des bestehenden Prototyps wurde so angepasst, dass diese leicht erweiterbar ist.
Weitere Funktionalitäten können so gut in das System integriert werden.